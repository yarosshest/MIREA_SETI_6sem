\section{Создание сети и настройка основных параметров устройства}

\subsection{Произведите базовую настройку маршрутизаторов}

\begin{verbatim}
hostname R1_Shestakov
no ip domain lookup
enable secret class
line console 0
password cisco
login
line vty 0 4
password cisco
login
service password-encryption
banner motd # You must be authorizeded! #
exit
copy running-config startup-config
\end{verbatim}

\subsection{Настройте базовые параметры каждого коммутатора}

\begin{verbatim}
hostname S1
no ip domain-lookup
enable secret class
line console 0
password cisco
login
line vty 0 15
password cisco
login
service password-encryption
banner motd # You must be authorizeded! #
interface range f0/1-3, f0/6-24, g0/1-2
shutdown
end
copy running-config startup-config
\end{verbatim}

\textbf{Вывод команды \texttt{show cdp neighbors} в этот момент
на R1\_ФАМИЛИЯ или R2 приводит к пустому списку. Дайте пояснение.}

Потому что интерфейсы маршрутизатора по умолчанию отключены.

\begin{image}
    \includegrph{img/img_2.png}
    \caption{Вывод команды show cdp neighbors}
\end{image}

\section{Настройка и проверка адресации IPv4 и IPv6 на R1\_ФАМИЛИЯ и R2}

\subsection{Настройте IP-адреса для обоих маршрутизаторов}

Включите одноадресную маршрутизацию IPv6.

\begin{verbatim}
// R*
ipv6 unicast-routing
\end{verbatim}

Настройте IP-адрес в соответствии с таблицей адресации.

\begin{verbatim}
// R1
interface g0/0/0
ip address 172.16.28.1 255.255.255.0
ipv6 address fe80::1 link-local
ipv6 address 2001:db8:acad:2::1/64
no shutdown
interface g0/0/1
ip address 192.168.1.1 255.255.255.0
ipv6 address fe80::1 link-local
ipv6 address 2001:db8:acad:1::1/64
no shutdown
interface lo1
ip address 10.1.0.1 255.255.255.0
ipv6 address fe80::1 link-local
ipv6 address 2001:db8:acad:10::1/64
no shutdown
interface lo2
ip address 209.165.200.225 255.255.255.224
ipv6 address fe80::1 link-local
ipv6 address 2001:db8:acad:209::1/64
no shutdown

// R2
interface g0/0/0
ip address 172.16.28.2 255.255.255.0
ipv6 address fe80::2 link-local
ipv6 address 2001:db8:acad:2::2/64
no shutdown
interface g0/0/1
ip address 192.168.1.2 255.255.255.0
ipv6 address fe80::2 link-local
ipv6 address 2001:db8:acad:1::2/64
no shutdown
interface lo1
ip address 10.2.0.1 255.255.255.0
ipv6 address fe80::2 link-local
ipv6 address 2001:db8:acad:11::2/64
no shutdown
interface lo2
ip address 209.165.200.193 255.255.255.224
ipv6 address fe80::2 link-local
ipv6 address 2001:db8:acad:210::1/64
no shutdown
\end{verbatim}

\subsection{Проверьте правильность IP-адресов}

Выполните команду, чтобы проверить назначения IPv4 интерфейсам.

\begin{verbatim}
show ip interface brief
\end{verbatim}

\begin{image}
    \includegrph{img/img.png}
    \caption{Вывод команды show ip interface brief}
\end{image}

Выполните команду, чтобы проверить назначения IPv6 интерфейсам.

\begin{verbatim}
show ipv6 interface brief
\end{verbatim}

\begin{image}
    \includegrph{img/img_1.png}
    \caption{Вывод команды show ipv6 interface brief}
\end{image}

\subsection{Сохраните конфигурацию}

Сохраните текущую конфигурацию в файл стартовой конфигурации
на обоих маршрутизаторах.

\begin{verbatim}
copy running-config startup-config
\end{verbatim}

\section{Настройка и проверка статической маршрутизации и
маршрутизации по умолчанию для IPv4 на R1\_ФАМИЛИЯ и R2}

\subsection{На R1\_ФАМИЛИЯ настройте статический маршрут к сети Loopback1 R2,
    используя адрес G0/0/1 R2 в качестве следующего перехода}

Используйте команду \texttt{ping}, чтобы убедиться,
что интерфейс G0/0/1 R2 доступен.

\begin{verbatim}
// R1
ping 192.168.1.2
\end{verbatim}

\begin{image}
    \includegrph{img/img_3.png}
    \caption{Вывод команды ping}
\end{image}

Настройте статический маршрут для сети Loopback1 R2 через адрес G0/0/1 R2.

\begin{verbatim}
// R1
ip route 10.2.0.0 255.255.255.0 192.168.1.2
\end{verbatim}

\subsection{На R1\_ФАМИЛИЯ настройте статический маршрут
по умолчанию через адрес G0/0/0 R2}

Используйте команду \texttt{ping}, чтобы убедиться,
что интерфейс G0/0/0 R2 доступен.

\begin{verbatim}
// R1
ping 172.16.28.2
\end{verbatim}

\begin{image}
    \includegrph{img/img_4.png}
    \caption{Вывод команды ping}
\end{image}

Настройте статический маршрут по умолчанию через адрес G0/0/0 R2.

\begin{verbatim}
// R1
ip route 0.0.0.0 0.0.0.0 172.16.1.2
\end{verbatim}

\subsection{На R1\_ФАМИЛИЯ настройте плавающий статический маршрут
по умолчанию через адрес G0/0/1 R2}

Настройте плавающий статический маршрут по умолчанию с AD 80 через адрес G0/1 R2.

\begin{verbatim}
// R1
ip route 0.0.0.0 0.0.0.0 192.168.1.2 80
\end{verbatim}

\subsection{На R2 настройте статический маршрут по умолчанию
через адрес G0/0/0 R1\_ФАМИЛИЯ}

Используйте команду \texttt{ping}, чтобы убедиться,
что интерфейс G0/0/0 R1\_ФАМИЛИЯ доступен.

\begin{verbatim}
// R2
ping 172.16.28.1
\end{verbatim}

\begin{image}
    \includegrph{img/img_5.png}
    \caption{Вывод команды ping}
\end{image}

Настройте статический маршрут по умолчанию через адрес G0/0/0 R1\_ФАМИЛИЯ.

\begin{verbatim}
// R2
ip route 0.0.0.0 0.0.0.0 172.16.28.1
\end{verbatim}

\subsection{Убедитесь, что маршруты работают}

Используйте команду \texttt{show ip route}, чтобы убедиться,
что в таблице маршрутизации R1\_ФАМИЛИЯ отображаются статические маршруты
и маршруты по умолчанию.

\begin{image}
    \includegrph{img/img_6.png}
    \caption{Вывод команды show ip route}
\end{image}

На R1\_ФАМИЛИЯ выполните команду \texttt{traceroute 10.2.0.1}.
Выходные данные должны показать, что следующий переход — 192.168.1.2.

\begin{image}
    \includegrph{img/img_7.png}
    \caption{Вывод команды traceroute 10.2.0.1}
\end{image}

На R1\_ФАМИЛИЯ выполните команду \texttt{traceroute 209.165.200.193}.
Выходные данные должны показать, что следующий переход — 172.16.X+1.2.

\begin{image}
    \includegrph{img/img_8.png}
    \caption{Вывод команды traceroute 209.165.200.193}
\end{image}

Выполните команду \texttt{shutdown} на R1\_ФАМИЛИЯ G0/0/0.

\begin{verbatim}
config terminal
interface g0/0/0
shutdown
end
\end{verbatim}

Покажите, что плавающий статический маршрут работает.
Выполните команду \texttt{show ip route static}.
Вы должны увидеть два статических маршрута.
Статический маршрут по умолчанию с AD
равным 80 и статическим маршрутом к сети 10.2.0.0/24 с AD равным 1.

\begin{image}
    \includegrph{img/img_9.png}
    \caption{Вывод команды show ip route static}
\end{image}

Демонстрация плавающего статического маршрута работает,
введите команду \texttt{traceroute 209.165.200.193}.
Вывод покажет следующий переход - 192.168.1.2.

\begin{image}
    \includegrph{img/img_10.png}
    \caption{Вывод команды traceroute 209.165.200.193}
\end{image}

Выполните команду \texttt{no shutdown} на R1\_ФАМИЛИЯ G0/0/0.

\begin{verbatim}
config terminal
interface g0/0/0
no shutdown
end
\end{verbatim}

\section{Настройка и проверка статической маршрутизации и
маршрутизации по умолчанию для IPv6 на R1\_ФАМИЛИЯ и R2}

\subsection{На R2 настройте статический маршрут к сети Loopback1 R1\_ФАМИЛИЯ,
    используя адрес G0/0/1 R1\_ФАМИЛИЯ в качестве следующего перехода}

Используйте команду \texttt{ping}, чтобы убедиться,
что интерфейс G0/0/1 R1\_ФАМИЛИЯ доступен.

\begin{verbatim}
// R2
ping 2001:db8:acad:1::1
\end{verbatim}

\begin{image}
    \includegrph{img/img_11.png}
    \caption{Вывод команды ping}
\end{image}

Настройте статический маршрут для сети Loopback1 R1\_ФАМИЛИЯ
через адрес G0/0/1 R1\_ФАМИЛИЯ.

\begin{verbatim}
// R2
ipv6 route 2001:db8:acad:10::/64 2001:db8:acad:1::1
\end{verbatim}

\subsection{На R2 настройте статический маршрут
по умолчанию через адрес G0/0/0 R1\_ФАМИЛИЯ}

Используйте команду \texttt{ping}, чтобы убедиться,
что интерфейс G0/0/0 R1\_ФАМИЛИЯ доступен.

\begin{verbatim}
// R2
ping 2001:db8:acad:2::1
\end{verbatim}

\begin{image}
    \includegrph{Screenshot from 2024-03-22 20-51-04}
    \caption{Вывод команды ping}
\end{image}

Настройте статический маршрут по умолчанию через адрес G0/0/0 R1\_ФАМИЛИЯ.

\begin{verbatim}
// R2
ipv6 route ::/0 2001:db8:acad:2::1
\end{verbatim}

\subsection{На R2 настройте плавающий статический маршрут
по умолчанию через адрес G0/0/1 R1\_ФАМИЛИЯ}

Настройте плавающий статический маршрут
по умолчанию с AD 80 через адрес G0/0/1 R2.

\begin{verbatim}
// R2
ipv6 route ::/0 2001:db8:acad:1::1 80
\end{verbatim}

\subsection{На R1\_ФАМИЛИЯ настройте статический маршрут
по умолчанию через адрес G0/0/0 R1\_ФАМИЛИЯ}

Используйте команду \texttt{ping}, чтобы убедиться,
что интерфейс G0/0/0 R2 доступен.

\begin{verbatim}
// R1
ping 2001:db8:acad:2::2
\end{verbatim}

\begin{image}
    \includegrph{Screenshot from 2024-03-22 20-55-16}
    \caption{Вывод команды ping}
\end{image}

Настройте статический маршрут по умолчанию через адрес G0/0/0 R2.

\begin{verbatim}
// R1
ipv6 route ::/0 2001:db8:acad:2::2
\end{verbatim}

\subsection{Убедитесь, что маршруты работают}

Используйте команду \texttt{show ipv6 route}, чтобы убедиться,
что таблица маршрутизации R2 отображает статические маршруты
и маршруты по умолчанию.

\begin{image}
    \includegrph{Screenshot from 2024-03-22 20-56-29}
    \caption{Вывод команды show ipv6 route}
\end{image}

На R2 выполните команду \texttt{traceroute 2001:db8:acad:10::1}.
Выходные данные должны показать, что следующий переход - 2001:db8:acad:1: :1.

\begin{image}
    \includegrph{Screenshot from 2024-03-22 20-56-47}
    \caption{Вывод команды traceroute 2001:db8:acad:10::1}
\end{image}

На R2 выполните команду \texttt{traceroute 2001:db8:acad:209::1}.
Выходные данные должны показать, что следующий переход - 2001:db8:acad:2::1.

\begin{image}
    \includegrph{Screenshot from 2024-03-22 20-57-12}
    \caption{Вывод команды traceroute 2001:db8:acad:209::1}
\end{image}

Выполните команду \texttt{shutdown} на R2 G0/0/0.

\begin{verbatim}
config terminal
interface g0/0/0
shutdown
end
\end{verbatim}

Покажите, что плавающий статический маршрут работает.
Выполните команду \texttt{show ipv6 route}.
Вы должны увидеть два статических маршрута.
Статический маршрут по умолчанию с AD 80
и статическим маршрутом в сеть 2001:db8:acad:10::/64 с AD 1.

\begin{image}
    \includegrph{Screenshot from 2024-03-22 20-59-43}
    \caption{Вывод команды show ipv6 route}
\end{image}

Наконец, продемонстрируйте, что плавающий статический маршрут работает,
выполнив команду \texttt{traceroute 2001:db8:acad:209::1}.
Следующий переход - 2001:db8:acad:1::1.

\begin{image}
    \includegrph{Screenshot from 2024-03-22 21-00-03}
    \caption{Вывод команды traceroute 2001:db8:acad:209::1}
\end{image}


\section{Ответы на контрольные вопросы}

\subsection{Опишите типы создания статических маршрутов.
Каков диапазон значений административного расстояния
и для настройки какого типа маршрута оно используется?}

Для настройки статических маршрутов в IPv4 и IPv6 доступны следующие варианты:

\begin{itemize}
    \item Стандартный статический маршрут.
    \item Статический маршрут по умолчанию.
    \item Плавающий статический маршрут.
    \item Сводный статический маршрут.
\end{itemize}

Команды глобальной конфигурации ip route и ipv6 route используются
для настройки статических маршрутов.
Административное расстояние:

\begin{itemize}
    \item Команда "distance" необязательна,
    но может быть использована для задания значения административного
    расстояния в диапазоне от 1 до 255.
    \item Обычно это применяется при настройке плавающего
    статического маршрута, устанавливая значение административного
    расстояния выше, чем для динамически определенного маршрута.
    \item Плавающие статические маршруты используются
    для обеспечения резервного маршрута в случае отказа основного
    статического или динамического маршрута.
\end{itemize}

\subsection{Дайте определение статическому маршруту по умолчанию.
Как определяется сеть назначения для статического IPv6 маршрута?}

Маршрут по умолчанию представляет собой статический маршрут,
который применяется ко всем пакетам.
Вместо того чтобы хранить маршруты для каждой сети в Интернете,
маршрутизаторы могут сохранять лишь один маршрут по умолчанию
для перенаправления трафика в любую сеть,
которая отсутствует в таблице маршрутизации.\par
Маршрут по умолчанию не требует специального сопоставления битов
слева адреса назначения IP. Он используется в случае,
когда ни один другой маршрут в таблице маршрутизации не соответствует
IP-адресу назначения пакета.

\subsection{В каком случае может потребоваться создание полностью заданного
статического маршрута и почему?
Какие параметры можно использовать для идентификации следующего перехода
в статическом маршруте?}

В случае полностью указанного статического маршрута указываются
как выходной интерфейс, так и IP-адрес следующего перехода.
Этот вид статического маршрута используется,
когда выходной интерфейс представляет собой интерфейс множественного доступа
и требуется явно указать следующий переход.
Следующий переход должен быть прямо подключен
к указанному выходному интерфейсу.
Использование выходного интерфейса необязательно, однако
обязательно указание адреса следующего перехода.\par
При настройке статического маршрута следующий переход можно определить
по IP-адресу, выходному интерфейсу или обоим параметрам одновременно.
В зависимости от того, как задан пункт назначения,
формируется один из трех типов статических маршрутов:

\begin{itemize}
    \item Маршрут следующего перехода --- указывается только IP-адрес
    следующего перехода.
    \item Статический маршрут с прямым подключением --- указывается
    только выходной интерфейс маршрутизатора.
    \item Полностью указанный статический маршрут --- указаны и IP-адрес
    следующего перехода, и выходной интерфейс.
\end{itemize}

\subsection{Каким образом можно создать статический маршрут
с прямым подключением?
Почему важно настраивать статический маршрут по умолчанию?}

При конфигурации статического маршрута альтернативным вариантом
является использование выходного интерфейса для определения адреса
следующего перехода.\par
Примечание: Обычно рекомендуется указывать адрес следующего перехода.
Статические маршруты с прямым подключением следует использовать только
с последовательными интерфейсами типа "точка-точка".\par
Когда маршрутизатор пытается маршрутизировать пакет,
и он не соответствует ни одному маршруту в таблице маршрутизации,
маршрутизатор обычно просто отбрасывает пакет.

\subsection{Для чего необходимо настраивать плавающий статический маршрут?
Что представляет из себя статический маршрут хостов?}

Плавающие статические маршруты представляют собой маршруты,
предназначенные для обеспечения резервного пути к основному статическому
или динамическому маршруту в случае сбоя соединения.
Они активируются только в том случае,
если основной маршрут становится недоступным.\par
Маршрут узла представляет собой IPv4-адрес с маской 32 бита
или IPv6-адрес с маской 128 бит.\par
Существуют три способа добавления маршрута узла в таблицу маршрутизации:

\begin{itemize}
    \item Автоматическая установка при настройке IP-адреса
    на маршрутизаторе (как показано на схемах).
    \item Настройка в качестве статического маршрута узла.
    \item Получение маршрута узла автоматически другими методами
    (рассматривается в последующих курсах).
\end{itemize}

\subsection{В каком случае в таблице маршрутизации появится
плавающий статический маршрут?
Для чего нужен суммарный статический маршрут?}

Для сокращения числа записей в таблице маршрутизации возможно
объединение нескольких статических маршрутов в один.
Это реализуется при соблюдении следующих условий:

\begin{itemize}
    \item Сети назначения должны быть смежными и могут быть объединены
    в один сетевой адрес.
    \item Все статические маршруты используют один
    и тот же выходной интерфейс или один
    и тот же IP-адрес следующего перехода.
\end{itemize}

\subsection{Что из себя представляет стандартный статический маршрут?
Почему для плавающего статического маршрута значение
административного расстояния (AD) должно быть больше,
    чем AD протокола динамической маршрутизации?}

Статический маршрут представляет собой постоянный и неизменный маршрут,
обычно вручную настроенный.\par
При добавлении статических маршрутов
в таблицу маршрутизации активируется маршрут,
указывающий на выходной интерфейс.\par
Далее создается плавающий статический маршрут через сеть ISP2.
Процесс его создания аналогичен обычному статическому маршруту по умолчанию,
за исключением указания административного расстояния.
Административное расстояние отражает степень надежности маршрута.
Значение административного расстояния статического маршрута равно единице,
что означает его абсолютный приоритет перед протоколами
динамической маршрутизации, у которых административное расстояние
обычно больше, за исключением локальных маршрутов,
у которых оно равно нулю.\par
Иногда администратор настраивает статический маршрут
к тому же самому месту назначения, которое уже известно
через протокол динамической маршрутизации, но с использованием другого пути.
Статический маршрут будет настроен с административным расстоянием, большим,
чем значение из протокола маршрутизации.
Если происходит сбой связи на пути, используемом протоколом динамической
маршрутизации, соответствующий маршрут этого протокола удаляется
из таблицы маршрутизации. Тогда статический маршрут станет единственным
источником и автоматически добавится в таблицу маршрутизации.
Это называется плавающим статическим маршрутом.

\subsection{Каким образом можно осуществить поиск и устранение неполадок,
    связанных со статическими маршрутами?
    Какой адрес и длина префикса используются при настройке
    статического маршрута IPv4 и IPv6 по умолчанию?}

Обратите внимание, что конфигурация статического маршрута
по умолчанию использует маску /0 для маршрутов по умолчанию IPv4
и префикс ::/0 для маршрутов по умолчанию IPv6.
Это означает, что ни один из битов IP-адреса назначения пакета
не должен совпадать с битами маршрута в таблице маршрутизации.
Маска /0 или префикс ::/0 указывает, что нет необходимости
в точном соответствии, и статический маршрут по умолчанию применяется
ко всем пакетам, пока не будет найдено более точное совпадение.