\section{Настройки безопасности коммутатора.}

\subsection{Релизация магистральных соединений 802.1Q.}

Настройте все магистральные порты Fa0/1 на обоих коммутаторах
для использования VLAN 333 в качестве native VLAN.

\begin{verbatim}
interface f0/1
switchport mode trunk
switchport trunk native vlan 333
\end{verbatim}

Убедитесь, что режим транкинга успешно настроен на всех коммутаторах
с помощью команды \texttt{show interface trunk} на обоих коммутаторах.

\img{img/img_3.png}{Результат комадны show interface trunk}
Отключить согласование DTP F0/1 на S1 и S2.

\begin{verbatim}
interface f0/1
switchport nonegotiate
\end{verbatim}

Проверьте с помощью команды \texttt{show interfaces}.

\img{img/img_4.png}{Результат комадны show interface}

\subsection{Настройка портов доступа}

На S1 настройте F0/5 и F0/6 в качестве портов доступа и свяжите их с VLAN 37.

\begin{verbatim}
interface range f0/5 - 6
switchport mode access
switchport access vlan 37
\end{verbatim}

На S2 настройте порт доступа Fa0/18 и свяжите его с VLAN 37.

\begin{verbatim}
interface f0/18
switchport mode access
switchport access vlan 37
\end{verbatim}

\subsection{Безопасность неиспользуемых портов коммутатора}

На S1 и S2 переместите неиспользуемые порты из VLAN 1 в VLAN 999
и отключите неиспользуемые порты.

\begin{verbatim}
interface range f0/2-4 , f0/7-24, g0/1-2  // S1
interface range f0/2-17 , f0/19-24, g0/1-2  // S2
switchport mode access
switchport access vlan 999
shutdown
\end{verbatim}

Убедитесь, что неиспользуемые порты отключены и связаны с VLAN 999,
введя команду \texttt{show interfaces status}.

\img{img/img_5.png}{Результат комадны show interface status}

\subsection{Документирование и реализация функций безопасности порта}

Интерфейсы F0/6 на S1 и F0/18 на S2 настроены как порты доступа.
На этом шаге вы также настроите безопасность портов
на этих двух портах доступа.

На S1 введите команду\texttt{show port-security interface f0/6}
для отображения настроек по умолчанию безопасности порта для интерфейса F0/6.

\img{img/img_6.png}{Результат комадны show port-security interface f0/6}

На S1 включите защиту порта на F0/6 со следующими настройками:

\begin{itemize}
    \item Максимальное количество записей MAC-адресов: 3
    \item Режим безопасности: restrict
    \item Aging time: 60 мин.
    \item Aging type: неактивный
\end{itemize}

\begin{verbatim}
interface f0/6
switchport port-security
switchport port-security maximum 3
switchport port-security violation restrict
switchport port-security aging time 60
switchport port-security aging type inactivity
\end{verbatim}

Проверьте настройки защиты порта (port-security) на S1 для интерфейса F0/6.
Далее просмотрите выходные данные команды \texttt{show port-security address}.

\img{img/img_7.png}{Результат комадны show port-security address}

\begin{verbatim}
show port-security interface f0/6
\end{verbatim}

\begin{verbatim}
show port-security address
\end{verbatim}


Включите безопасность порта для F0/18 на S2.
Настройте каждый активный порт доступа таким образом,
чтобы он автоматически добавлял адреса МАС, изученные на этом порту,
в текущую конфигурацию.

\begin{verbatim}
interface f0/18
switchport port-security
switchport port-security mac-address sticky
\end{verbatim}

Настройте следующие параметры безопасности порта на S2 F0/18:

\begin{itemize}
    \item Максимальное количество записей MAC-адресов: 2
    \item Тип безопасности: Protect
    \item Aging time: 60 мин.
\end{itemize}

\begin{verbatim}
interface f0/18
switchport port-security aging time 60
switchport port-security maximum 2
switchport port-security violation protect
\end{verbatim}

Проверьте настройки защиты порта (port-security)
на S2 для интерфейса F0/18. Далее просмотрите выходные
данные команды \texttt{show port-security address}.

\begin{verbatim}
show port-security interface f0/18
\end{verbatim}

\begin{verbatim}
show port-security address
\end{verbatim}

\img{img/img_8.png}{Результат комадны show port-security address}

\subsection{Реализовать безопасность DHCP snooping}

На S2 включите DHCP snooping и настройте DHCP snooping во VLAN X+10.

\begin{verbatim}
ip dhcp snooping
ip dhcp snooping vlan 37
\end{verbatim}

Настройте магистральные порты на S2 как доверенные порты.

\begin{verbatim}
interface f0/1
ip dhcp snooping trust
\end{verbatim}

Ограничьте ненадежный порт Fa0/18 на S2 пятью DHCP-пакетами в секунду.

\begin{verbatim}
interface f0/18
ip dhcp snooping limit rate 5
\end{verbatim}

Проверьте DHCP Snooping на S2 с помощью команды show ip dhcp snooping.

\begin{verbatim}
show ip dhcp snooping
\end{verbatim}

\img{img/img_9.png}{Результат комадны show ip dhcp snooping}


В командной строке на PC-B освободите, а затем обновите IP-адрес.

\begin{verbatim}
C:\Users\Student> ipconfig /release
C:\Users\Student> ipconfig /renew
\end{verbatim}

\img{img/img_10.png}{Результат комадны ipconfig}

Проверьте привязку отслеживания DHCP
с помощью команды \texttt{show ip dhcp snooping binding}.

\img{img/img_11.png}{Результат комадны show ip dhcp snooping binding}

\subsection{Реализация PortFast и BPDU Guard}

Настройте PortFast на всех портах доступа,
которые используются на обоих коммутаторах.

\begin{verbatim}
interface range f0/5 - 6  // S1
interface f0/18  // S2
spanning-tree portfast
\end{verbatim}

Включите защиту BPDU на портах доступа VLAN 13 для S1 и S2,
подключенных к PC-A и PC-B.

\begin{verbatim}
interface f0/6  // S1
interface f0/18  // S2
spanning-tree bpduguard enable
\end{verbatim}

Убедитесь, что защита BPDU и PortFast включены на соответствующих портах
с помощью команды \texttt{show spanning-tree interface f0/6 detail}.

\begin{verbatim}
show spanning-tree interface f0/6 detail
\end{verbatim}

\img{img/img_12.png}{Результат комадны show spanning-tree interface f0/6 detail}

\img{img/img_13.png}{Результат комадны spanning-tree interface f0/18 detail}




