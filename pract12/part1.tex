\section{Создание сети и настройка основных параметров устройства}

\subsection{Настройте основные параметры на каждом
маршрутизаторе и коммутаторе}

\begin{verbatim}
no ip domain lookup
service password-encryption
enable secret class
line console 0
password cisco
login
logging synchronous
line vty 0 4
password cisco
login
logging synchronous
banner motd # You must be authorizeded! #

/* Switch1 */
hostname S1
interface vlan 1
no shutdown
interface range f0/1-24, g0/1-2
shutdown
interface f0/5
no shutdown
switchport mode access
switchport access vlan 1
interface f0/6
no shutdown
switchport mode access
switchport access vlan 1
/* Switch */

/* Switch3 */
hostname S3
interface vlan 1
no shutdown
interface range f0/1-24, g0/1-2
shutdown
interface f0/5
no shutdown
switchport mode access
switchport access vlan 1
interface f0/18
no shutdown
switchport mode access
switchport access vlan 1
/* Switch */

/* Routre */
// R1
hostname R1
interface vlan 1
no shutdown
interface g 0/1
no shutdown
ip address 172.30.10.1 255.255.255.0
description R1 gb 0/1
interface se 0/0/0
no shutdown
ip address 10.1.1.1 255.255.255.252
description R1 se 0/1
clock rate 64000

// R2_ФАМИЛИЯ
hostname R2_Shestakov
interface g 0/0
no shutdown
ip address 209.165.228.1 255.255.255.0
description R2 gb 0/0
interface se 0/0/0
no shutdown
ip address 10.1.1.2 255.255.255.252
description R2 se 0
clock rate 64000
interface se 0/0/1
no shutdown
ip address 10.2.2.2 255.255.255.252
description R2 se 1
clock rate 64000


// R3
hostname R3
interface vlan 1
no shutdown
interface g 0/1
no shutdown
ip address 172.30.30.1 255.255.255.0
description R3 gb 0/1
interface se 0/0/1
no shutdown
ip address 10.2.2.1 255.255.255.252
description R3 se 0/0/1
clock rate 64000
/* Routre */


end
copy running-config startup-config
\end{verbatim}

\subsection{Проверка связи}

На данный момент компьютеры не могут отправлять друг другу эхо-запросы.

Каждая рабочая станция должна иметь возможность проводить
эхо-тестирование присоединенного маршрутизатора.
При неудачном выполнении эхо-запросов выполните поиск и устранение неполадок.

Маршрутизаторы должны успешно отправлять эхо-запросы друг другу.
При неудачномвыполнении эхо-запросов выполните поиск и устранение неполадок

\begin{image}
    \includegrph{img/img.png}
    \caption{Выполнение эхо-запросов}
\end{image}

\section{Настройка и проверка маршрутизации RIPv2}

\subsection{Настройте маршрутизацию по протоколу RIPv2}
Настройте протокол RIPv2 на маршрутизаторе R1
в качестве протоколамаршрутизации и проинформируйте
об этом соответствующие подключенные сети.

\begin{verbatim}
config t
router rip
version 2
passive-interface g0/1
network 172.30.0.0
network 10.0.0.0
\end{verbatim}

Настройте протокол RIPv2 на маршрутизаторе R3 и воспользуйтесь
командой network, чтобы добавить соответствующие сети
и предотвратить обновления маршрутизации в интерфейсе локальной сети.

\begin{verbatim}
config t
router rip
version 2
passive-interface g0/1
network 172.30.0.0
network 10.0.0.0
\end{verbatim}

Настройте протокол RIPv2 на маршрутизаторе R2\_ФАМИЛИЯ и воспользуйтесь
командой network,чтобы добавить соответствующие подключенные сети.
Не объявляйте сеть 209.165.228.0.

\begin{verbatim}
config t
router rip
version 2
passive-interface g0/0
network 10.0.0.0
\end{verbatim}

\subsection{Проверьте текущее состояние сети}

Состояние двух последовательных каналов можно легко проверить
с помощью команды \texttt{show ip interface brief}
на маршрутизаторе R2\_ФАМИЛИЯ.

\begin{image}
    \includegrph{img/img_1.png}
    \caption{Вывод команды show ip interface brief}
\end{image}

Проверьте наличие подключения между компьютерами.

\textbf{Успешно ли отправляется эхо-запрос от узла PC-A на PC-B? Почему?}

Нет, R2 не объявлен маршрут к PC-B.

\textbf{Успешно ли проходит эхо-запрос с PC-A на PC-С? Почему?}

Нет, у R1 и R3 нет маршрутов к удаленным сетям,
а у R2 неправильно указаны два маршрута балансировки нагрузки
с равной стоимостью к подсети 172.30.0.0/16.

\textbf{Успешно ли проходит эхо-запрос с узла PC-С на PC-B? Почему?}

Нет, R2 не объявлен маршрут к PC-B.

\textbf{Успешно ли проходит эхо-запрос с узла PC-С на PC-А? Почему?}

Нет, у R1 и R3 нет маршрутов к удаленным сетям,
а у R2 неправильно указаны два маршрута балансировки нагрузки
с равной стоимостью к подсети 172.30.0.0/16.

Убедитесь в том, что протокол RIPv2 активирован на маршрутизаторах.
Чтобы проверить это, можно воспользоваться командами \texttt{debug ip rip},
\texttt{show ip protocols} и \texttt{show run}.

\begin{image}
    \includegrph{img/img_2.png}
    \caption{Вывод команды show ip protocols и debug ip rip}
\end{image}

\begin{image}
    \includegrph{img/img_3.png}
    \caption{Вывод команды debug ip rip}
\end{image}

\textbf{Какие сведения подтверждают работу RIPv2
при выполнении команды debug ip rip на маршрутизаторе R2\_ФАМИЛИЯ?}

RIP: sending v2 updates to 224.0.0.9 via Serial 0/0/0 (10.1.1.2).
RIP: sending v2 updates to 224.0.0.9 via Serial 0/0/1 (10.2.2.2).

Изучив выходные данные отладки, в командной строке привилегированного
режима выполните команду \textbf{undebug all}.

\textbf{Какие сведения подтверждают работу RIPv2
при выполнении команды show run на маршрутизаторе R3?}

router rip
version 2

\begin{image}
    \includegrph{img/img_4.png}
    \caption{Вывод команды show run}
\end{image}

Отключите автоматическое суммирование маршрутов.

Локальные сети, подключенные к маршрутизаторам R1 и R3,
состоят из «разорванных» сетей. Маршрутизатор R2\_ФАМИЛИЯ отображает
в таблице маршрутизации два пути к сети 172.30.0.0/16,
имеющие одинаковую стоимость. Маршрутизатор R2\_ФАМИЛИЯ отображает
только адрес главной классовой сети 172.30.0.0,
но не отображает подсети этой сети.

\begin{verbatim}
// R2
show ip route
\end{verbatim}

\begin{image}
    \includegrph{img/img_5.png}
    \caption{Вывод команды show ip route}
\end{image}

Маршрутизатор R1 отображает только собственную подсеть
для сети 172.30.10.0/24. В таблице маршрутизации R1 нет маршрута
для подсети 172.30.30.0/24 маршрутизатора R3.

\begin{verbatim}
// R1
show ip route
\end{verbatim}

\begin{image}
    \includegrph{img/img_6.png}
    \caption{Вывод команды show ip route}
\end{image}

Маршрутизатор R3 отображает только собственную подсеть
для сети 172.30.30.0/24. В таблице маршрутизации R3 нет маршрута
для подсетей 172.30.10.0/24 маршрутизатора R1.

\begin{verbatim}
// R3
show ip route
\end{verbatim}

\begin{image}
    \includegrph{img/img_7.png}
    \caption{Вывод команды show ip route}
\end{image}

Чтобы определить маршруты, полученные в обновлениях RIP от маршрутизатора R3,
используйте команду debug ip rip на маршрутизаторе R2\_ФАМИЛИЯ.
Укажите их далее.

172.30.0.0/16

\subsection{Отключите автоматическое объединение}

Для отключения автоматического суммирования
в RIPv2 используется команда \texttt{no auto-summary}.
Отключите автоматическое суммирование на всех маршрутизаторах.
Маршрутизаторы больше не суммируют маршруты на границах главной
классовой сети.

\begin{verbatim}
router rip
no auto-summary
\end{verbatim}

Чтобы очистить таблицу маршрутизации, используйте команду clear ip route *.

\begin{verbatim}
clear ip route *
\end{verbatim}

Изучите таблицы маршрутизации. Не забывайте, что после очистки таблиц
маршрутизации потребуется некоторое время для выравнивания их содержимого.
Подсети LAN, подключенные к маршрутизаторам R1 и R3,
должны быть включены во все три таблицы маршрутизации.

\begin{verbatim}
show ip route
\end{verbatim}

\begin{image}
    \includegrph{img/img_8.png}
    \caption{Вывод команды show ip route}
\end{image}

Чтобы проверить обновления RIP, на маршрутизаторе
R2\_ФАМИЛИЯ используйте команду \texttt{debug ip rip}.

Через 60 секунд выполните команду \texttt{no debug ip rip}.

\begin{image}
    \includegrph{img/img_9.png}
    \caption{Вывод команды debug ip rip}
\end{image}

\textbf{Какие маршруты содержатся в обновлениях RIP, принятых от R3?}

172.30.30.0/24

\textbf{Включены ли маски подсети в обновления маршрутизации?}

Да

\subsection{Настройка и перераспределение маршрута
по умолчанию для доступа к Интернету}

На маршрутизаторе R2\_ФАМИЛИЯ создайте статический маршрут
к сети 0.0.0.0 0.0.0.0 с помощью команды \texttt{ip route}.
В результате весь трафик с неизвестным адресом назначения будет пересылаться
на компьютер PC-B с адресом 209.165.228.2,
моделируя Интернет путем настройки шлюза "<последней надежды">
на маршрутизаторе R2\_ФАМИЛИЯ.

\begin{verbatim}
ip route 0.0.0.0 0.0.0.0 209.165.228.2
\end{verbatim}

Маршрутизатор R2\_ФАМИЛИЯ объявит маршрут для других маршрутизаторов,
если команда \texttt{default-information originate} будет добавлена
в его конфигурацию RIP.

\begin{verbatim}
router rip
default-information originate
\end{verbatim}

\subsection{Проверка конфигурации маршрутизации}

Просмотрите таблицу маршрутизации маршрутизатора R1.

\begin{verbatim}
show ip route
\end{verbatim}

\begin{image}
    \includegrph{img/img_10.png}
    \caption{Вывод команды show ip route}
\end{image}

\textbf{Как на основании таблицы маршрутизации можно определить, что сеть,
    разбитая на подсети и используемая маршрутизаторами R1 и R3,
    имеет путь для интернет-трафика?}

Существует шлюз последней узла,
и маршрут по умолчанию отображается в таблице как изучаемый с помощью RIP.

Просмотрите таблицу маршрутизации на R2\_ФАМИЛИЯ.

\begin{image}
    \includegrph{img/img_11.png}
    \caption{Вывод команды show ip route}
\end{image}

\textbf{Каким образом путь для интернет-трафика появился
в таблице маршрутизации маршрутизатора R2\_ФАМИЛИЯ?}

R2 имеет статический маршрут по умолчанию к 0.0.0.0 через 209.165.201.2,
который напрямую подключен к G0/0.

\subsection{Проверьте подключение}

Смоделируйте отправку трафика в Интернет,
отправив эхо-запросы от узла PC-A и PC-C в сеть 209.165.228.2.

\textbf{Успешно ли выполнена проверка связи?}

Да

Убедитесь в том, что узлы в разбитой на подсети сети могут достичь друг друга.
Для этого выполните эхо-запрос между узлами PC-A и PC-C.

\textbf{Успешно ли выполнена проверка связи?}

Да

\begin{image}
    \includegrph{img/img_12.png}
    \caption{Вывод команды ping}
\end{image}

\section{Ответы на контрольные вопросы}

\subsection{Дайте характеристику механизмам пересылки пакетов.
Опишите все возможные источники получения маршрутов
в таблице маршрутизации}

Существует несколько механизмов маршрутизации,
которые маршрутизатор использует для построения
и поддержания в актуальном состоянии своей таблицы маршрутизации.
В общем случае при построении таблицы маршрутизации маршрутизатор
применяет комбинацию следующих методов маршрутизации:

\begin{itemize}
    \item Прямое соединение --- это маршрут, который является локальным
    по отношению к маршрутизатору;
    \item Статическая маршрутизация --- это такие маршруты к сетям получателям,
    которые администратор сети вручную вносит в таблицу маршрутизации;
    \item Маршрутизация по умолчанию --- маршрутизатор может посылать
    весь трафик или часть трафика, не описанного в таблице маршрутизации,
    по специальному маршруту, так называемому маршруту по умолчанию;
    \item Динамическая маршрутизация --- автоматическое отслеживание
    изменения в топологии сети.
\end{itemize}

И хотя каждый из этих методов имеет свои преимущества и недостатки,
они не являются взаимоисключающими.\par
Первым источником является программное обеспечение стека TCP/IP.
Вторым источником появления записи в таблице является администратор,
непосредственно формирующий запись с помощью некоторой системной утилиты.
И наконец, третьим источником записей могут быть протоколы маршрутизации,
такие как RIP или OSPF.

\subsection{Дайте определение понятию "<административное расстояние"> (AD).
Какое AD используется протоколом RIP по умолчанию и как его посмотреть?}

Административное расстояние --- это число,
величина которого определяется источником задаваемого маршрута.
Меньшее административное расстояние означает более надежный источник.\par
По умолчанию для протокола RIP (в том числе и для RIPv2)
используется административное расстояние (AD) равное 120.\par
В Cisco IOS командой \texttt{show ip protocols} можно просмотреть
текущие настройки протоколов маршрутизации, включая информацию о значениях AD.

\subsection{В каких случаях целесообразно настроить динамическую маршрутизацию?
Дайте определение понятию "<метрика маршрута">}

Динамическую маршрутизацию целесообразно настраивать
в случаях изменчивой топологии сети, крупных и распределенных сетях,
для автоматизации обнаружения соседних устройств
и обмена маршрутной информацией, упрощения управления сетью
и предотвращения человеческих ошибок,
а также для автоматического обнаружения и обхода недоступных маршрутов.\par
Метрика маршрута --- это числовое значение,
используемое протоколом маршрутизации для оценки "дороговизны"
или "качества" маршрута. Она обычно отражает различные аспекты,
такие как пропускная способность, задержка, стоимость и надежность маршрута.

\subsection{Проведите краткую сравнительную характеристику статической
и динамической маршрутизации на основе нескольких критериев.
Какие бывают протоколы динамической маршрутизации
    (опишите категории и приведите примеры)?}

Статическая маршрутизация требует ручной настройки каждого маршрута
и не обновляется автоматически при изменениях в сети.
Динамическая маршрутизация автоматически обновляет маршруты
на основе информации от соседних маршрутизаторов,
обеспечивая адаптацию к изменениям в сети.
Динамическая маршрутизация более эффективна в использовании ресурсов сети,
так как обменивается только необходимой информацией о маршрутах,
в то время как статическая маршрутизация требует хранения полной таблицы
маршрутов на каждом маршрутизаторе. Динамическая маршрутизация также обладает
встроенной отказоустойчивостью,
что позволяет ей быстро реагировать на недоступность маршрутов,
в отличие от статической маршрутизации,
которая не обладает такой функциональностью.

\begin{enumerate}
    \item Автономная система (AS) --- это набор маршрутизаторов имеющих
    единые правила маршрутизации и управляемых одной технической
    администрацией и работающих на одном из ШПЗ протоколов.
    \item Внутренние протоколы --- маршрутизация внутри AS
    \item Внешние протоколы --- маршрутизация между AS
    \item Distance-Vector Protocol (протокол вектора расстояния)
    \begin{itemize}
        \item это протокол на базе вектора расстояния
        \item маршрутизатор не знает всего пути до сети назначения;
        \item знает направление (вектор);
        \item знает расстояние (число переходов);
    \end{itemize}
    Обмен маршрутами выполняется периодически,
    даже если не было изменений в топологии. Маршрутизатор:
    \begin{itemize}
        \item отправляет свои маршруты соседям
        \item получает от соседей сведения об известных им маршрутах.
    \end{itemize}
    \item Link State Routing Protocol (протоколы состояния канала)
    \begin{itemize}
        \item протоколы на основе состояния канала;
        \item маршрутизатор создает собственную топологию сети;
        \item маршрутизаторы обмениваются сведениями
        о непосредственно подключенных и активных каналах
        \item обновления отправляются периодически;
        \item обновления отправляются триггерно,
        только в случае изменения топологии сети
        (изменении состояния канала).
    \end{itemize}
    \item RIP (Routing Information Protocol).
    Примеры: RIP – 1988, RIPv2 – 1994, RIPng – 1997.
    \begin{itemize}
        \item это distance-vector routing protocol;
        \item используется число переходов (hops) в качестве метрики;
        \item является classfull протоколом;
        \item допускает максимально 15 переходов (hops);
        \item обновления рассылаются через 30 секунд;
        \item административное расстояние AD = 120
    \end{itemize}
    \item RIPv2
    \begin{itemize}
        \item это distance-vector routing protocol;
        \item является classless протоколом;
        \item поддерживает VLSM и CIDR (Classless Inter Domain Routing);
        \item использует число переходов (hops) в качестве метрики;
        \item административное расстояние AD=120
        \item обновления отправляются через 30 секунд;
        \item - поддерживает аутентификацию.
    \end{itemize}
    \item RIPng
    \begin{itemize}
        \item поддерживает IPv6
    \end{itemize}
    \item EIGRP (Enhanced Interior Gateway Routing Protocol)
    --- улучшенный IGRP. Примеры: IGRP – 1985, EIGRP – 1992.
    Протокол разработан компанией Cisco на основе IGRP
    \begin{itemize}
        \item поддерживает VLSM и CIDR
        \item является гибридным протоколом сочетает качества
        distance-vector и link-state протоколов;
        \item обеспечивает быструю сходимость сети;
        \item эффективно использует полосу пропускания,
        рассылая частичные обновления;
        \item использует специальную таблицу соседей;
        \item использует специальную таблицу топологии (содержит все маршруты);
        \item использует алгоритм DUAL для заполнения routing table;
        \item использует составную метрику
        (полоса пропускания, загрузка, задержка, надежность)
    \end{itemize}
    \item OSPF (Open Shortest Path First) - протокол выбора кротчайшего пути.
    Примеры: OSPFv2 – 1991, OSPFv3 – 1999
    \begin{itemize}
        \item поддерживается специальная база данных о соседях
        (состояние каналов);
        \item используется алгоритм SPF (Shortest Path First)
        для формирования записей в таблице маршрутизации;
        \item метрика --- это ширина полосы пропускания;
        \item поддерживает VLSM и CIDR;
        \item поддерживает аутентификацию.
    \end{itemize}
\end{enumerate}

\subsection{Для чего нужны протоколы динамической маршрутизации?
Какие компоненты включают в себя протоколы динамической маршрутизации?}

Протоколы динамической маршрутизации используются для передачи информации
о том, какие сети в настоящее время подключены к каждому из маршрутизаторов.
Маршрутизаторы общаются, используя протоколы маршрутизации.

Протоколы динамической маршрутизации включают в себя следующие компоненты:

\begin{enumerate}
    \item \textbf{Структуры данных}. Протоколы маршрутизации обычно используют
    для своих операций таблицы или базы данных.
    Данная информация хранится в ОЗУ.
    \item \textbf{Сообщения протокола маршрутизации}.
    Протоколы маршрутизации используют различные типы сообщений
    для обнаружения соседних маршрутизаторов,
    обмена информацией о маршрутах и выполнения других задач,
    связанных с получением точной информации о сети.
    \item \textbf{Алгоритм}. Алгоритм представляет собой определённый
    список действий, используемых для выполнения задачи.
    Протоколы маршрутизации используют алгоритмы,
    упрощающие обмен данных маршрутизации и определение оптимального пути.
\end{enumerate}

\subsection{Как вычисляется метрика для протоколов RIP, OSPF и EIGRP?
Как работает распределение нагрузки
при использовании динамической маршрутизации?}

Метрика может основываться на одной характеристике маршрута или на нескольких его характеристиках. Ниже описаны наиболее часто используемые метрики.

\begin{itemize}
    \item Ширина полосы пропускания (Bandwidth).
    Максимальный объем данных за единицу времени,
    которые могут передаваться по каналу
    (обычно канал Ethernet 10 Мбит/с предпочтительнее вьщеленной
    линии 64 Кбит/с).
    \item Задержка (Delay). Промежуток времени, необходимый пакету
    для прохождения по каналу всего пути от источника до получателя.
    Задержка зависит от полосы пропускания промежуточных каналов,
    очередей на портах каждого промежуточного маршрутизатора,
    переполнений в сети и физического расстояния.
    \item Нагрузка (Load). Уровень активности сетевых компонентов, таких,
    как маршрутизатор или канал.
    \item Надежность (Reliability). Обычно под надежностью понимается
    вероятность ошибки в канале сети.
    \item Количество переходов (Hop count). Количество маршрутизаторов,
    через которые должен пройти пакет перед тем как он достигнет своего
    пункта назначения. Каждый маршрутизатор, через который проходят
    данные, рассматривается как один переход.
    Если количество переходов маршрута равно четырем, то это означает,
    что пакет должен пройти через четыре маршрутизатора перед тем,
    как он достигнет пункта назначения. Если к пункту назначения
    существует несколько маршрутов, то маршрутизатор выбирает маршрут
    с наименьшим количеством переходов.
    \item Оценка (Cost). Произвольное значение, назначаемое сетевым
    администратором; обычно оно основывается на полосе пропускания,
    финансовых затратах или на других измерениях.
\end{itemize}

\subsection{Опишите назначение кодов C, L и S в таблице маршрутизации.
В каких случаях используется протокол BGP?}

Код определяет, каким образом был получен маршрут:
\begin{itemize}
    \item L --- указывает адрес, назначенный интерфейсу маршрутизатора.
    Данный код позволяет маршрутизатору быстро определить,
    что полученный пакет предназначен для интерфейса, а не для пересылки;
    \item C --- определяет сеть с прямым подключением;
    \item S --- определяет статический маршрут,
    созданный для достижения конкретной сети;
\end{itemize}

Протокол BGP предназначен для обмена информацией о достижимости подсетей
между автономными системами (АС), то есть группами маршрутизаторов
под единым техническим управлением, использующими протокол
внутридоменной маршрутизации для определения маршрутов внутри себя
и протокол междоменной маршрутизации для определения маршрутов
доставки пакетов в другие АС.

\subsection{Что является недостатком динамической маршрутизации?
Что представляет из себя "<пассивный интерфейс">?}

Недостатками динамической маршрутизации являются:

\begin{itemize}
    \item реализация может предполагать высокий уровень сложности;
    \item требуется знание дополнительных команд для реализации;
    \item менее безопасна (для обеспечения высокого уровня безопасности
    требуется дополнительная настройка);
    \item маршрут зависит от текущей топологии;
    \item требуются дополнительные ресурсы центрального процессора,
    оперативного запоминающего устройства и полосы пропускания канала.
\end{itemize}

Пассивный интерфейс только принимает данные.
Проблемы разделения канала между интерфейсами здесь нет.