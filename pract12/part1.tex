\section{Построение сети и проверка связи}
\subsection{Создайте сеть согласно топологии.}
Подключите устройства, как показано в топологии, и подсоедините необходимые кабели.

\subsection{Настройте узлы ПК.}

\subsection{ Выполните инициализацию и перезагрузку маршрутизатора и коммутаторов. }



\subsection{Произведите базовую настройку маршрутизаторов.}

\begin{verbatim}
    /* Начало */
    no ip domain lookup
    service password-encryption
    enable secret class
    line console 0
    password cisco
    login
    logging synchronous
    line vty 0 15
    password class
    login
    logging synchronous
    banner motd # You must be authorizeded! #
    /* Начало */

    /* R1 */
    hostname R1
    interface g 0/1
    ip address 192.168.1.1 255.255.255.0
    no shutdown
    interface se 0/0/0
    ip address 10.1.1.1 255.255.255.252
    no shutdown
    clock rate 128000
    /* R1 */

    /* R2_Shestakov */
    hostname R2_Shestakov
    interface se 0/0/0
    ip address 10.1.1.2 255.255.255.252
    no shutdown
    interface se 0/0/1
    ip address 10.2.2.2 255.255.255.252
    no shutdown
    clock rate 128000
    interface Lo1
    ip address 209.165.227.225 255.255.255.224
    /* R2_Shestakov */

    /* R3 */
    hostname R3
    interface g 0/1
    ip address 192.168.1.3 255.255.255.0
    no shutdown
    interface se 0/0/1
    ip address 10.2.2.1 255.255.255.252
    no shutdown
    /* R3 */

    /* Конец */
    end
    copy running-config startup-config
    /* Конец */
\end{verbatim}

\subsection{ Настройте базовые параметры каждого коммутатора.}

\begin{verbatim}
    /* Начало */
    no ip domain lookup
    service password-encryption
    enable secret class
    line console 0
    password cisco
    login
    logging synchronous
    line vty 0 15
    password class
    login
    logging synchronous
    banner motd # You must be authorizeded! #
    /* Начало */

    /* Switch1 */
    hostname S1
    interface vlan 1
    ip address 192.168.1.11 255.255.255.0
    ip default-gateway 192.168.1.1
    no shutdown
    /* Switch1 */

    /* Switch3 */
    hostname S3
    interface vlan 1
    ip address 192.168.1.13 255.255.255.0
    ip default-gateway 192.168.1.3
    no shutdown
    /* Switch3 */

    /* Конец */
    end
    copy running-config startup-config
    /* Конец */
\end{verbatim}

\subsection{Проверьте подключение между PC-A и PC-C.}

Отправьте ping-запрос с компьютера PC-A на компьютер PC-C. Удалось липолучить ответ? \textbf{Да}
Если команды ping завершились неудачно и связь установить не удалось, исправьте ошибки
в основных настройках устройства.


\begin{image}
    \includegrph{img/img.png}
    \caption{Выполнение эхо-запросов}
\end{image}

\subsection{Настройте маршрутизацию.}
Настройте RIP версии 2 на всех маршрутизаторах.
Добавьте в процесс RIP все сети, кроме
209.165.X+200.224/27.

\begin{verbatim}
    /* R1 R3 */
    config t
    router rip
    version 2
    no auto-summary
    do sh ip route connected
    network 10.0.0.0
    network 192.168.1.0
    /* R1 R3 */

    /* R2 */
    config t
    router rip
    version 2
    no auto-summary
    do sh ip route connected
    network 10.0.0.0
    /* R2 */
\end{verbatim}

Настройте маршрут по умолчанию на маршрутизаторе R2\_ФАМИЛИЯ с использованием
Lo1 в качестве интерфейса выхода в сеть 209.165.X+200.224/27.

\begin{verbatim}
    /* R2 */
    config t
    ip route 0.0.0.0 0.0.0.0 loopback 1
    /* R2 */
\end{verbatim}

На маршрутизаторе R2\_ФАМИЛИЯ используйте следующие команды для перераспределения
маршрута по умолчанию в процесс RIP.

\begin{verbatim}
    /* R2 */
    config t
    router rip
    default-information originate
    /* R2 */
\end{verbatim}

\subsection{Проверьте подключение}

Необходимо получить ответ на ping-запросы с компьютера PC-A от каждого интерфейса на
маршрутизаторах R1, R2\_ФАМИЛИЯ и R3, а также от компьютера PC-C. Удалось ли получить
все ответы?
Если команды ping завершились неудачно и связь установить не удалось, исправьте ошибки
в основных настройках устройства.

\begin{image}
    \includegrph{img/img_1.png}
    \caption{Эхо-запрос от компьютера PC-A на интерфейс G0/1 маршрутизатора R1}
\end{image}

\begin{image}
    \includegrph{img/img_2.png}
    \caption{Эхо-запрос от компьютера PC-A на интерфейс S0/0/0 маршрутизатора R1}
\end{image}

\begin{image}
    \includegrph{img/img_3.png}
    \caption{Эхо-запрос от компьютера PC-A на интерфейс S0/0/0 маршрутизатора R2}
\end{image}

\begin{image}
    \includegrph{img/img_4.png}
    \caption{Эхо-запрос от компьютера PC-A на интерфейс S0/0/1 маршрутизатора R2}
\end{image}

\begin{image}
    \includegrph{img/img_5.png}
    \caption{Эхо-запрос от компьютера PC-A на интерфейс Lo1 маршрутизатора R2}
\end{image}

\begin{image}
    \includegrph{img/img_6.png}
    \caption{Эхо-запрос от компьютера PC-A на интерфейс G0/1 маршрутизатора R1}
\end{image}

\begin{image}
    \includegrph{img/img_7.png}
    \caption{Эхо-запрос от компьютера PC-A на интерфейс S0/0/1 маршрутизатора R1}
\end{image}

\begin{image}
    \includegrph{img/img_8.png}
    \caption{Эхо-запрос от компьютера PC-C на интерфейс G0/1 маршрутизатора R1}
\end{image}

\begin{image}
    \includegrph{img/img_9.png}
    \caption{Эхо-запрос от компьютера PC-C на интерфейс S0/0/0 маршрутизатора R1}
\end{image}

\begin{image}
    \includegrph{img/img_10.png}
    \caption{Эхо-запрос от компьютера PC-C на интерфейс S0/0/0 маршрутизатора R2}
\end{image}

\begin{image}
    \includegrph{img/img_11.png}
    \caption{Эхо-запрос от компьютера PC-C на интерфейс S0/0/1 маршрутизатора R2}
\end{image}

\begin{image}
    \includegrph{img/img_12.png}
    \caption{Эхо-запрос от компьютера PC-C на интерфейс Lo1 маршрутизатора R2}
\end{image}

\begin{image}
    \includegrph{img/img_13.png}
    \caption{Эхо-запрос от компьютера PC-C на интерфейс G0/1 маршрутизатора R1}
\end{image}

\begin{image}
    \includegrph{img/img_14.png}
    \caption{Эхо-запрос от компьютера PC-C на интерфейс S0/0/1 маршрутизатора R1}
\end{image}

\section{Настройка обеспечения избыточности на первом хопе
с помощью HSRP}

Даже если топология спроектирована с учетом избыточности (два маршрутизатора и два коммутатора
в одной сети LAN), оба компьютера, PC-A и PC-C, необходимо настраивать с одним адресом шлюза.
PC-A использует R1, а PC-C — R3. В случае сбоя на одном из этих маршрутизаторов или интерфейсов
маршрутизаторов компьютер может потерять подключение к сети Интернет.
В части 2 вам предстоит изучить поведение сети до и после настройки протокола HSRP. Для этого вам
понадобится определить путь, по которому проходят пакеты, чтобы достичь loopback-адрес на
R2\_ФАМИЛИЯ.

\subsection{Определите путь интернет-трафика для PC-A и PC-C.}
В командной строке на PC-A введите команду tracert для loopback-адреса 209.165.X+200.225
на маршрутизаторе R2\_ФАМИЛИЯ.
Какой путь прошли пакеты от PC-A до 209.165.X+200.225?

\begin{image}
    \includegrph{img/img_15.png}
    \caption{tracert от PC-A до 209.165.X+200.225 }
\end{image}

В командной строке на PC-С введите команду tracert для loopback-адреса 209.165.X+200.225
на маршрутизаторе R2\_ФАМИЛИЯ.
Какой путь прошли пакеты от PC-C до 209.165.X+200.225?
\begin{image}
    \includegrph{img/img_16.png}
    \caption{tracert от PC-С до 209.165.X+200.225 }
\end{image}

\subsection{Запустите сеанс эхо-тестирования на PC-A и разорвите соединение между S1 и R1.}
В командной строке на PC-A введите команду ping –t для адреса 209.165.X+200.225
на маршрутизаторе R2\_ФАМИЛИЯ. Убедитесь, что окно командной строки открыто

\begin{image}
    \includegrph{img/img_17.png}
    \caption{ping –t от PC-А до 209.165.X+200.225 }
\end{image}

В процессе эхо-тестирования отсоедините кабель Ethernet от интерфейса F0/5 на S1.
Отключение интерфейса F0/5 на S1 приведет к тому же результату.
Что произошло с трафиком эхо-запросов?

\begin{image}
    \includegrph{img/img_18.png}
    \caption{Отключение интерфейса F0/5 на S1 PC-A}
\end{image}

Какими были бы результате при повторении шагов 2a и 2b на компьютере PC-C и коммутаторе S3?
Такой же результат.

Повторно подсоедините кабели Ethernet к интерфейсу F0/5 или включите интерфейс F0/5 на S1 и S3, соответственно.
Повторно отправьте эхо-запросы на 209.165.X+200.225 с компьютеров PC-A и PC-C, чтобы убедиться в том, что подключение восстановлено.

\begin{image}
    \includegrph{img/img_19.png}
    \caption{Отключение интерфейса F0/5 на S1 PC-C}
\end{image}

Повторно подсоедините кабели Ethernet к интерфейсу F0/5 или включите интерфейс F0/5 на S1
и S3, соответственно. Повторно отправьте эхо-запросы на 209.165.X+200.225 с компьютеров
PC-A и PC-C, чтобы убедиться в том, что подключение восстановлено.

\subsection{Настройте HSRP на R1 и R3.}
В этом шаге вам предстоит настроить HSRP и изменить адрес шлюза по умолчанию на компьютерах
PC-A, PC-C, S1 и коммутаторе S2 на виртуальный IP-адрес для HSRP. R1 назначается активным
маршрутизатором с помощью команды приоритета HSRP.

\begin{verbatim}
    /* R1 */
    interface g 0/1
    standby version 2
    standby 1 ip 192.168.1.254
    standby 1 priority 150
    standby 1 preempt
    /* R1 */

    /* R3 */
    interface g 0/1
    standby version 2
    standby 1 ip 192.168.1.254
    /* R3 */

    /* Конец */
    end
    copy running-config startup-config
    /* Конец */
\end{verbatim}

Проверьте HSRP, выполнив команду show standby на R1 и R3.

\begin{image}
    \includegrph{img/img_20.png}
    \caption{show standby на R1 и R3}
\end{image}

Используя указанные выходные данные, ответьте на следующие вопросы:
Какой маршрутизатор является активным?
Маршрутизатор R1
Какой MAC-адрес используется для виртуального IP-адреса?
0000.0C9F.F001
Какой IP-адрес и приоритет используются для резервного маршрутизатора?
Priority 100
192.168.1.254

Используйте команду show standby brief на R1 и R3, чтобы просмотреть сводку состояния HSRP.

\begin{image}
    \includegrph{img/img_21.png}
    \caption{show standby brief на R1 и R3}
\end{image}
Измените адрес шлюза по умолчанию для PC-A, PC-C, S1 и S3. Какой адрес следует использовать?
PC-A
ip 192.168.1.31
mask 255.255.255.0
gw 192.168.1.1 -> 192.168.1.254

PC-C
ip 192.168.1.33
mask 255.255.255.0
gw 192.168.1.3 -> 192.168.1.254

\begin{verbatim}
    /* S1 */
    ip default-gateway 192.168.1.254
    /* S1 */

    /* S3 */
    ip default-gateway 192.168.1.254
    /* S3 */
\end{verbatim}

Проверьте новые настройки.
Отправьте эхо-запрос с PC-A и с PC-C на loopback-адрес маршрутизатора R2\_ФАМИЛИЯ.
Успешно ли выполнены эхо-запросы?

\begin{image}
    \includegrph{img/img_22.png}
    \caption{эхо-запрос с PC-A на loopback-адрес маршрутизатора R2\_ФАМИЛИЯ}
\end{image}

\begin{image}
    \includegrph{img/img_23.png}
    \caption{эхо-запрос с PC-C на loopback-адрес маршрутизатора R2\_ФАМИЛИЯ}
\end{image}

\subsection{Запустите сеанс эхо-тестирования на PC-A и разорвите соединение
с коммутатором, подключенным к активному маршрутизатору HSRP (R1).}

В командной строке на PC-A введите команду ping –t для адреса 209.165.X+200.225
на маршрутизаторе R2. Убедитесь, что окно командной строки открыто.

Во время отправки эхо-запроса отсоедините кабель Ethernet от интерфейса F0/5 на коммутаторе S1 или
выключите интерфейс F0/5.
\begin{image}
    \includegrph{img/img_24.png}
    \caption{ping –t для адреса 209.165.X+200.225
    на маршрутизаторе R2}
\end{image}

Что произошло с трафиком эхо-запросов?
Компьютер PC-A по-прежнему может отправлять эхо-запросы на интерфейс маршрутизатора R2.

\subsection{Проверьте настройки HSRP на маршрутизаторах R1 и R3.}
Выполните команду show standby brief на маршрутизаторах R1 и R3.
\begin{image}
    \includegrph{img/img_25.png}
    \caption{show standby brief на маршрутизаторах R1 и R3}
\end{image}

show standby brief на маршрутизаторах R1 и R3

Повторно подключите кабель, соединяющий коммутатор и маршрутизатор, или включите интерфейс F0/5.
Какой маршрутизатор теперь является активным?
Поясните ответ.
\begin{image}
    \includegrph{img/img_26.png}
    \caption{show standby brief на маршрутизаторах R1 и R3}
\end{image}

R1 стал активным маршрутизатором, поскольку вытеснение включено и имеет более высокий приоритет
\section{Изменение приоритетов HSRP.}

Измените приоритет HSRP на 200 на маршрутизаторе R3. Какой маршрутизатор является
активным?

\begin{image}
    \includegrph{img/img_27.png}
    \caption{ HSRP на 200 на маршрутизаторе R3}
\end{image}

Выполните команду, чтобы сделать активным маршрутизатор R3 без изменения приоритета.
Используйте команду show, чтобы убедиться, что R3 является активным маршрутизатором.
\begin{image}
    \includegrph{img/img_28.png}
    \caption{ HSRP на 200 на маршрутизаторе R3}
\end{image}

\section{Ответы на контрольные вопросы}

\subsection{Дайте характеристику механизмам пересылки пакетов.
Опишите все возможные источники получения маршрутов
в таблице маршрутизации}

Существует несколько механизмов маршрутизации,
которые маршрутизатор использует для построения
и поддержания в актуальном состоянии своей таблицы маршрутизации.
В общем случае при построении таблицы маршрутизации маршрутизатор
применяет комбинацию следующих методов маршрутизации:

\begin{itemize}
    \item Прямое соединение --- это маршрут, который является локальным
    по отношению к маршрутизатору;
    \item Статическая маршрутизация --- это такие маршруты к сетям получателям,
    которые администратор сети вручную вносит в таблицу маршрутизации;
    \item Маршрутизация по умолчанию --- маршрутизатор может посылать
    весь трафик или часть трафика, не описанного в таблице маршрутизации,
    по специальному маршруту, так называемому маршруту по умолчанию;
    \item Динамическая маршрутизация --- автоматическое отслеживание
    изменения в топологии сети.
\end{itemize}

И хотя каждый из этих методов имеет свои преимущества и недостатки,
они не являются взаимоисключающими.\par
Первым источником является программное обеспечение стека TCP/IP.
Вторым источником появления записи в таблице является администратор,
непосредственно формирующий запись с помощью некоторой системной утилиты.
И наконец, третьим источником записей могут быть протоколы маршрутизации,
такие как RIP или OSPF.

\subsection{Дайте определение понятию "<административное расстояние"> (AD).
Какое AD используется протоколом RIP по умолчанию и как его посмотреть?}

Административное расстояние --- это число,
величина которого определяется источником задаваемого маршрута.
Меньшее административное расстояние означает более надежный источник.\par
По умолчанию для протокола RIP (в том числе и для RIPv2)
используется административное расстояние (AD) равное 120.\par
В Cisco IOS командой \texttt{show ip protocols} можно просмотреть
текущие настройки протоколов маршрутизации, включая информацию о значениях AD.

\subsection{В каких случаях целесообразно настроить динамическую маршрутизацию?
Дайте определение понятию "<метрика маршрута">}

Динамическую маршрутизацию целесообразно настраивать
в случаях изменчивой топологии сети, крупных и распределенных сетях,
для автоматизации обнаружения соседних устройств
и обмена маршрутной информацией, упрощения управления сетью
и предотвращения человеческих ошибок,
а также для автоматического обнаружения и обхода недоступных маршрутов.\par
Метрика маршрута --- это числовое значение,
используемое протоколом маршрутизации для оценки "дороговизны"
или "качества" маршрута. Она обычно отражает различные аспекты,
такие как пропускная способность, задержка, стоимость и надежность маршрута.

\subsection{Проведите краткую сравнительную характеристику статической
и динамической маршрутизации на основе нескольких критериев.
Какие бывают протоколы динамической маршрутизации
    (опишите категории и приведите примеры)?}

Статическая маршрутизация требует ручной настройки каждого маршрута
и не обновляется автоматически при изменениях в сети.
Динамическая маршрутизация автоматически обновляет маршруты
на основе информации от соседних маршрутизаторов,
обеспечивая адаптацию к изменениям в сети.
Динамическая маршрутизация более эффективна в использовании ресурсов сети,
так как обменивается только необходимой информацией о маршрутах,
в то время как статическая маршрутизация требует хранения полной таблицы
маршрутов на каждом маршрутизаторе. Динамическая маршрутизация также обладает
встроенной отказоустойчивостью,
что позволяет ей быстро реагировать на недоступность маршрутов,
в отличие от статической маршрутизации,
которая не обладает такой функциональностью.

\begin{enumerate}
    \item Автономная система (AS) --- это набор маршрутизаторов имеющих
    единые правила маршрутизации и управляемых одной технической
    администрацией и работающих на одном из ШПЗ протоколов.
    \item Внутренние протоколы --- маршрутизация внутри AS
    \item Внешние протоколы --- маршрутизация между AS
    \item Distance-Vector Protocol (протокол вектора расстояния)
    \begin{itemize}
        \item это протокол на базе вектора расстояния
        \item маршрутизатор не знает всего пути до сети назначения;
        \item знает направление (вектор);
        \item знает расстояние (число переходов);
    \end{itemize}
    Обмен маршрутами выполняется периодически,
    даже если не было изменений в топологии. Маршрутизатор:
    \begin{itemize}
        \item отправляет свои маршруты соседям
        \item получает от соседей сведения об известных им маршрутах.
    \end{itemize}
    \item Link State Routing Protocol (протоколы состояния канала)
    \begin{itemize}
        \item протоколы на основе состояния канала;
        \item маршрутизатор создает собственную топологию сети;
        \item маршрутизаторы обмениваются сведениями
        о непосредственно подключенных и активных каналах
        \item обновления отправляются периодически;
        \item обновления отправляются триггерно,
        только в случае изменения топологии сети
        (изменении состояния канала).
    \end{itemize}
    \item RIP (Routing Information Protocol).
    Примеры: RIP – 1988, RIPv2 – 1994, RIPng – 1997.
    \begin{itemize}
        \item это distance-vector routing protocol;
        \item используется число переходов (hops) в качестве метрики;
        \item является classfull протоколом;
        \item допускает максимально 15 переходов (hops);
        \item обновления рассылаются через 30 секунд;
        \item административное расстояние AD = 120
    \end{itemize}
    \item RIPv2
    \begin{itemize}
        \item это distance-vector routing protocol;
        \item является classless протоколом;
        \item поддерживает VLSM и CIDR (Classless Inter Domain Routing);
        \item использует число переходов (hops) в качестве метрики;
        \item административное расстояние AD=120
        \item обновления отправляются через 30 секунд;
        \item - поддерживает аутентификацию.
    \end{itemize}
    \item RIPng
    \begin{itemize}
        \item поддерживает IPv6
    \end{itemize}
    \item EIGRP (Enhanced Interior Gateway Routing Protocol)
    --- улучшенный IGRP. Примеры: IGRP – 1985, EIGRP – 1992.
    Протокол разработан компанией Cisco на основе IGRP
    \begin{itemize}
        \item поддерживает VLSM и CIDR
        \item является гибридным протоколом сочетает качества
        distance-vector и link-state протоколов;
        \item обеспечивает быструю сходимость сети;
        \item эффективно использует полосу пропускания,
        рассылая частичные обновления;
        \item использует специальную таблицу соседей;
        \item использует специальную таблицу топологии (содержит все маршруты);
        \item использует алгоритм DUAL для заполнения routing table;
        \item использует составную метрику
        (полоса пропускания, загрузка, задержка, надежность)
    \end{itemize}
    \item OSPF (Open Shortest Path First) - протокол выбора кротчайшего пути.
    Примеры: OSPFv2 – 1991, OSPFv3 – 1999
    \begin{itemize}
        \item поддерживается специальная база данных о соседях
        (состояние каналов);
        \item используется алгоритм SPF (Shortest Path First)
        для формирования записей в таблице маршрутизации;
        \item метрика --- это ширина полосы пропускания;
        \item поддерживает VLSM и CIDR;
        \item поддерживает аутентификацию.
    \end{itemize}
\end{enumerate}

\subsection{Для чего нужны протоколы динамической маршрутизации?
Какие компоненты включают в себя протоколы динамической маршрутизации?}

Протоколы динамической маршрутизации используются для передачи информации
о том, какие сети в настоящее время подключены к каждому из маршрутизаторов.
Маршрутизаторы общаются, используя протоколы маршрутизации.

Протоколы динамической маршрутизации включают в себя следующие компоненты:

\begin{enumerate}
    \item \textbf{Структуры данных}. Протоколы маршрутизации обычно используют
    для своих операций таблицы или базы данных.
    Данная информация хранится в ОЗУ.
    \item \textbf{Сообщения протокола маршрутизации}.
    Протоколы маршрутизации используют различные типы сообщений
    для обнаружения соседних маршрутизаторов,
    обмена информацией о маршрутах и выполнения других задач,
    связанных с получением точной информации о сети.
    \item \textbf{Алгоритм}. Алгоритм представляет собой определённый
    список действий, используемых для выполнения задачи.
    Протоколы маршрутизации используют алгоритмы,
    упрощающие обмен данных маршрутизации и определение оптимального пути.
\end{enumerate}

\subsection{Как вычисляется метрика для протоколов RIP, OSPF и EIGRP?
Как работает распределение нагрузки
при использовании динамической маршрутизации?}

Метрика может основываться на одной характеристике маршрута или на нескольких его характеристиках. Ниже описаны наиболее часто используемые метрики.

\begin{itemize}
    \item Ширина полосы пропускания (Bandwidth).
    Максимальный объем данных за единицу времени,
    которые могут передаваться по каналу
    (обычно канал Ethernet 10 Мбит/с предпочтительнее вьщеленной
    линии 64 Кбит/с).
    \item Задержка (Delay). Промежуток времени, необходимый пакету
    для прохождения по каналу всего пути от источника до получателя.
    Задержка зависит от полосы пропускания промежуточных каналов,
    очередей на портах каждого промежуточного маршрутизатора,
    переполнений в сети и физического расстояния.
    \item Нагрузка (Load). Уровень активности сетевых компонентов, таких,
    как маршрутизатор или канал.
    \item Надежность (Reliability). Обычно под надежностью понимается
    вероятность ошибки в канале сети.
    \item Количество переходов (Hop count). Количество маршрутизаторов,
    через которые должен пройти пакет перед тем как он достигнет своего
    пункта назначения. Каждый маршрутизатор, через который проходят
    данные, рассматривается как один переход.
    Если количество переходов маршрута равно четырем, то это означает,
    что пакет должен пройти через четыре маршрутизатора перед тем,
    как он достигнет пункта назначения. Если к пункту назначения
    существует несколько маршрутов, то маршрутизатор выбирает маршрут
    с наименьшим количеством переходов.
    \item Оценка (Cost). Произвольное значение, назначаемое сетевым
    администратором; обычно оно основывается на полосе пропускания,
    финансовых затратах или на других измерениях.
\end{itemize}

\subsection{Опишите назначение кодов C, L и S в таблице маршрутизации.
В каких случаях используется протокол BGP?}

Код определяет, каким образом был получен маршрут:
\begin{itemize}
    \item L --- указывает адрес, назначенный интерфейсу маршрутизатора.
    Данный код позволяет маршрутизатору быстро определить,
    что полученный пакет предназначен для интерфейса, а не для пересылки;
    \item C --- определяет сеть с прямым подключением;
    \item S --- определяет статический маршрут,
    созданный для достижения конкретной сети;
\end{itemize}

Протокол BGP предназначен для обмена информацией о достижимости подсетей
между автономными системами (АС), то есть группами маршрутизаторов
под единым техническим управлением, использующими протокол
внутридоменной маршрутизации для определения маршрутов внутри себя
и протокол междоменной маршрутизации для определения маршрутов
доставки пакетов в другие АС.

\subsection{Что является недостатком динамической маршрутизации?
Что представляет из себя "<пассивный интерфейс">?}

Недостатками динамической маршрутизации являются:

\begin{itemize}
    \item реализация может предполагать высокий уровень сложности;
    \item требуется знание дополнительных команд для реализации;
    \item менее безопасна (для обеспечения высокого уровня безопасности
    требуется дополнительная настройка);
    \item маршрут зависит от текущей топологии;
    \item требуются дополнительные ресурсы центрального процессора,
    оперативного запоминающего устройства и полосы пропускания канала.
\end{itemize}

Пассивный интерфейс только принимает данные.
Проблемы разделения канала между интерфейсами здесь нет.