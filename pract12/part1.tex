\section{Построение сети и проверка связи}
\subsection{Создайте сеть согласно топологии.}
Подключите устройства, как показано в топологии, и подсоедините необходимые кабели.

\subsection{Настройте узлы ПК.}

\subsection{ Выполните инициализацию и перезагрузку маршрутизатора и коммутаторов. }



\subsection{Произведите базовую настройку маршрутизаторов.}

\begin{verbatim}
    /* Начало */
    no ip domain lookup
    service password-encryption
    enable secret class
    line console 0
    password cisco
    login
    logging synchronous
    line vty 0 15
    password class
    login
    logging synchronous
    banner motd # You must be authorizeded! #
    /* Начало */

    /* R1 */
    hostname R1
    interface g 0/1
    ip address 192.168.1.1 255.255.255.0
    no shutdown
    interface se 0/0/0
    ip address 10.1.1.1 255.255.255.252
    no shutdown
    clock rate 128000
    /* R1 */

    /* R2_Shestakov */
    hostname R2_Shestakov
    interface se 0/0/0
    ip address 10.1.1.2 255.255.255.252
    no shutdown
    interface se 0/0/1
    ip address 10.2.2.2 255.255.255.252
    no shutdown
    clock rate 128000
    interface Lo1
    ip address 209.165.227.225 255.255.255.224
    /* R2_Shestakov */

    /* R3 */
    hostname R3
    interface g 0/1
    ip address 192.168.1.3 255.255.255.0
    no shutdown
    interface se 0/0/1
    ip address 10.2.2.1 255.255.255.252
    no shutdown
    /* R3 */

    /* Конец */
    end
    copy running-config startup-config
    /* Конец */
\end{verbatim}

\subsection{ Настройте базовые параметры каждого коммутатора.}

\begin{verbatim}
    /* Начало */
    no ip domain lookup
    service password-encryption
    enable secret class
    line console 0
    password cisco
    login
    logging synchronous
    line vty 0 15
    password class
    login
    logging synchronous
    banner motd # You must be authorizeded! #
    /* Начало */

    /* Switch1 */
    hostname S1
    interface vlan 1
    ip address 192.168.1.11 255.255.255.0
    ip default-gateway 192.168.1.1
    no shutdown
    /* Switch1 */

    /* Switch3 */
    hostname S3
    interface vlan 1
    ip address 192.168.1.13 255.255.255.0
    ip default-gateway 192.168.1.3
    no shutdown
    /* Switch3 */

    /* Конец */
    end
    copy running-config startup-config
    /* Конец */
\end{verbatim}

\subsection{Проверьте подключение между PC-A и PC-C.}

Отправьте ping-запрос с компьютера PC-A на компьютер PC-C. Удалось липолучить ответ? \textbf{Да}
Если команды ping завершились неудачно и связь установить не удалось, исправьте ошибки
в основных настройках устройства.


\begin{image}
    \includegrph{img/img.png}
    \caption{Выполнение эхо-запросов}
\end{image}

\subsection{Настройте маршрутизацию.}
Настройте RIP версии 2 на всех маршрутизаторах.
Добавьте в процесс RIP все сети, кроме
209.165.X+200.224/27.

\begin{verbatim}
    /* R1 R3 */
    config t
    router rip
    version 2
    no auto-summary
    do sh ip route connected
    network 10.0.0.0
    network 192.168.1.0
    /* R1 R3 */

    /* R2 */
    config t
    router rip
    version 2
    no auto-summary
    do sh ip route connected
    network 10.0.0.0
    /* R2 */
\end{verbatim}

Настройте маршрут по умолчанию на маршрутизаторе R2\_ФАМИЛИЯ с использованием
Lo1 в качестве интерфейса выхода в сеть 209.165.X+200.224/27.

\begin{verbatim}
    /* R2 */
    config t
    ip route 0.0.0.0 0.0.0.0 loopback 1
    /* R2 */
\end{verbatim}

На маршрутизаторе R2\_ФАМИЛИЯ используйте следующие команды для перераспределения
маршрута по умолчанию в процесс RIP.

\begin{verbatim}
    /* R2 */
    config t
    router rip
    default-information originate
    /* R2 */
\end{verbatim}

\subsection{Проверьте подключение}

Необходимо получить ответ на ping-запросы с компьютера PC-A от каждого интерфейса на
маршрутизаторах R1, R2\_ФАМИЛИЯ и R3, а также от компьютера PC-C. Удалось ли получить
все ответы?
Если команды ping завершились неудачно и связь установить не удалось, исправьте ошибки
в основных настройках устройства.

\begin{image}
    \includegrph{img/img_1.png}
    \caption{Эхо-запрос от компьютера PC-A на интерфейс G0/1 маршрутизатора R1}
\end{image}

\begin{image}
    \includegrph{img/img_2.png}
    \caption{Эхо-запрос от компьютера PC-A на интерфейс S0/0/0 маршрутизатора R1}
\end{image}

\begin{image}
    \includegrph{img/img_3.png}
    \caption{Эхо-запрос от компьютера PC-A на интерфейс S0/0/0 маршрутизатора R2}
\end{image}

\begin{image}
    \includegrph{img/img_4.png}
    \caption{Эхо-запрос от компьютера PC-A на интерфейс S0/0/1 маршрутизатора R2}
\end{image}

\begin{image}
    \includegrph{img/img_5.png}
    \caption{Эхо-запрос от компьютера PC-A на интерфейс Lo1 маршрутизатора R2}
\end{image}

\begin{image}
    \includegrph{img/img_6.png}
    \caption{Эхо-запрос от компьютера PC-A на интерфейс G0/1 маршрутизатора R1}
\end{image}

\begin{image}
    \includegrph{img/img_7.png}
    \caption{Эхо-запрос от компьютера PC-A на интерфейс S0/0/1 маршрутизатора R1}
\end{image}

\begin{image}
    \includegrph{img/img_8.png}
    \caption{Эхо-запрос от компьютера PC-C на интерфейс G0/1 маршрутизатора R1}
\end{image}

\begin{image}
    \includegrph{img/img_9.png}
    \caption{Эхо-запрос от компьютера PC-C на интерфейс S0/0/0 маршрутизатора R1}
\end{image}

\begin{image}
    \includegrph{img/img_10.png}
    \caption{Эхо-запрос от компьютера PC-C на интерфейс S0/0/0 маршрутизатора R2}
\end{image}

\begin{image}
    \includegrph{img/img_11.png}
    \caption{Эхо-запрос от компьютера PC-C на интерфейс S0/0/1 маршрутизатора R2}
\end{image}

\begin{image}
    \includegrph{img/img_12.png}
    \caption{Эхо-запрос от компьютера PC-C на интерфейс Lo1 маршрутизатора R2}
\end{image}

\begin{image}
    \includegrph{img/img_13.png}
    \caption{Эхо-запрос от компьютера PC-C на интерфейс G0/1 маршрутизатора R1}
\end{image}

\begin{image}
    \includegrph{img/img_14.png}
    \caption{Эхо-запрос от компьютера PC-C на интерфейс S0/0/1 маршрутизатора R1}
\end{image}

\section{Настройка обеспечения избыточности на первом хопе
с помощью HSRP}

Даже если топология спроектирована с учетом избыточности (два маршрутизатора и два коммутатора
в одной сети LAN), оба компьютера, PC-A и PC-C, необходимо настраивать с одним адресом шлюза.
PC-A использует R1, а PC-C — R3. В случае сбоя на одном из этих маршрутизаторов или интерфейсов
маршрутизаторов компьютер может потерять подключение к сети Интернет.
В части 2 вам предстоит изучить поведение сети до и после настройки протокола HSRP. Для этого вам
понадобится определить путь, по которому проходят пакеты, чтобы достичь loopback-адрес на
R2\_ФАМИЛИЯ.

\subsection{Определите путь интернет-трафика для PC-A и PC-C.}
В командной строке на PC-A введите команду tracert для loopback-адреса 209.165.X+200.225
на маршрутизаторе R2\_ФАМИЛИЯ.
Какой путь прошли пакеты от PC-A до 209.165.X+200.225?

\begin{image}
    \includegrph{img/img_15.png}
    \caption{tracert от PC-A до 209.165.X+200.225 }
\end{image}

В командной строке на PC-С введите команду tracert для loopback-адреса 209.165.X+200.225
на маршрутизаторе R2\_ФАМИЛИЯ.
Какой путь прошли пакеты от PC-C до 209.165.X+200.225?
\begin{image}
    \includegrph{img/img_16.png}
    \caption{tracert от PC-С до 209.165.X+200.225 }
\end{image}

\subsection{Запустите сеанс эхо-тестирования на PC-A и разорвите соединение между S1 и R1.}
В командной строке на PC-A введите команду ping –t для адреса 209.165.X+200.225
на маршрутизаторе R2\_ФАМИЛИЯ. Убедитесь, что окно командной строки открыто

\begin{image}
    \includegrph{img/img_17.png}
    \caption{ping –t от PC-А до 209.165.X+200.225 }
\end{image}

В процессе эхо-тестирования отсоедините кабель Ethernet от интерфейса F0/5 на S1.
Отключение интерфейса F0/5 на S1 приведет к тому же результату.
Что произошло с трафиком эхо-запросов?

\begin{image}
    \includegrph{img/img_18.png}
    \caption{Отключение интерфейса F0/5 на S1 PC-A}
\end{image}

Какими были бы результате при повторении шагов 2a и 2b на компьютере PC-C и коммутаторе S3?
Такой же результат.

Повторно подсоедините кабели Ethernet к интерфейсу F0/5 или включите интерфейс F0/5 на S1 и S3, соответственно.
Повторно отправьте эхо-запросы на 209.165.X+200.225 с компьютеров PC-A и PC-C, чтобы убедиться в том, что подключение восстановлено.

\begin{image}
    \includegrph{img/img_19.png}
    \caption{Отключение интерфейса F0/5 на S1 PC-C}
\end{image}

Повторно подсоедините кабели Ethernet к интерфейсу F0/5 или включите интерфейс F0/5 на S1
и S3, соответственно. Повторно отправьте эхо-запросы на 209.165.X+200.225 с компьютеров
PC-A и PC-C, чтобы убедиться в том, что подключение восстановлено.

\subsection{Настройте HSRP на R1 и R3.}
В этом шаге вам предстоит настроить HSRP и изменить адрес шлюза по умолчанию на компьютерах
PC-A, PC-C, S1 и коммутаторе S2 на виртуальный IP-адрес для HSRP. R1 назначается активным
маршрутизатором с помощью команды приоритета HSRP.

\begin{verbatim}
    /* R1 */
    interface g 0/1
    standby version 2
    standby 1 ip 192.168.1.254
    standby 1 priority 150
    standby 1 preempt
    /* R1 */

    /* R3 */
    interface g 0/1
    standby version 2
    standby 1 ip 192.168.1.254
    /* R3 */

    /* Конец */
    end
    copy running-config startup-config
    /* Конец */
\end{verbatim}

Проверьте HSRP, выполнив команду show standby на R1 и R3.

\begin{image}
    \includegrph{img/img_20.png}
    \caption{show standby на R1 и R3}
\end{image}

Используя указанные выходные данные, ответьте на следующие вопросы:
Какой маршрутизатор является активным?
Маршрутизатор R1
Какой MAC-адрес используется для виртуального IP-адреса?
0000.0C9F.F001
Какой IP-адрес и приоритет используются для резервного маршрутизатора?
Priority 100
192.168.1.254

Используйте команду show standby brief на R1 и R3, чтобы просмотреть сводку состояния HSRP.

\begin{image}
    \includegrph{img/img_21.png}
    \caption{show standby brief на R1 и R3}
\end{image}
Измените адрес шлюза по умолчанию для PC-A, PC-C, S1 и S3. Какой адрес следует использовать?
PC-A
ip 192.168.1.31
mask 255.255.255.0
gw 192.168.1.1 -> 192.168.1.254

PC-C
ip 192.168.1.33
mask 255.255.255.0
gw 192.168.1.3 -> 192.168.1.254

\begin{verbatim}
    /* S1 */
    ip default-gateway 192.168.1.254
    /* S1 */

    /* S3 */
    ip default-gateway 192.168.1.254
    /* S3 */
\end{verbatim}

Проверьте новые настройки.
Отправьте эхо-запрос с PC-A и с PC-C на loopback-адрес маршрутизатора R2\_ФАМИЛИЯ.
Успешно ли выполнены эхо-запросы?

\begin{image}
    \includegrph{img/img_22.png}
    \caption{эхо-запрос с PC-A на loopback-адрес маршрутизатора R2\_ФАМИЛИЯ}
\end{image}

\begin{image}
    \includegrph{img/img_23.png}
    \caption{эхо-запрос с PC-C на loopback-адрес маршрутизатора R2\_ФАМИЛИЯ}
\end{image}

\subsection{Запустите сеанс эхо-тестирования на PC-A и разорвите соединение
с коммутатором, подключенным к активному маршрутизатору HSRP (R1).}

В командной строке на PC-A введите команду ping –t для адреса 209.165.X+200.225
на маршрутизаторе R2. Убедитесь, что окно командной строки открыто.

Во время отправки эхо-запроса отсоедините кабель Ethernet от интерфейса F0/5 на коммутаторе S1 или
выключите интерфейс F0/5.
\begin{image}
    \includegrph{img/img_24.png}
    \caption{ping –t для адреса 209.165.X+200.225
    на маршрутизаторе R2}
\end{image}

Что произошло с трафиком эхо-запросов?
Компьютер PC-A по-прежнему может отправлять эхо-запросы на интерфейс маршрутизатора R2.

\subsection{Проверьте настройки HSRP на маршрутизаторах R1 и R3.}
Выполните команду show standby brief на маршрутизаторах R1 и R3.
\begin{image}
    \includegrph{img/img_25.png}
    \caption{show standby brief на маршрутизаторах R1 и R3}
\end{image}

show standby brief на маршрутизаторах R1 и R3

Повторно подключите кабель, соединяющий коммутатор и маршрутизатор, или включите интерфейс F0/5.
Какой маршрутизатор теперь является активным?
Поясните ответ.
\begin{image}
    \includegrph{img/img_26.png}
    \caption{show standby brief на маршрутизаторах R1 и R3}
\end{image}

R1 стал активным маршрутизатором, поскольку вытеснение включено и имеет более высокий приоритет
\section{Изменение приоритетов HSRP.}

Измените приоритет HSRP на 200 на маршрутизаторе R3. Какой маршрутизатор является
активным?

\begin{image}
    \includegrph{img/img_27.png}
    \caption{ HSRP на 200 на маршрутизаторе R3}
\end{image}

Выполните команду, чтобы сделать активным маршрутизатор R3 без изменения приоритета.
Используйте команду show, чтобы убедиться, что R3 является активным маршрутизатором.
\begin{image}
    \includegrph{img/img_28.png}
    \caption{ HSRP на 200 на маршрутизаторе R3}
\end{image}

\section{Ответы на контрольные вопросы}

\subsection{Для чего необходимо резервирование маршрутизаторов? Опишите преимущества протокола HSRP.}

Одним из способов предотвращения единой точки отказа на шлюзе по умолчанию является внедрение виртуального маршрутизатора. Чтобы реализовать этот тип резервирования маршрутизатора, несколько маршрутизаторов настраиваются для совместной работы, чтобы создать иллюзию одного маршрутизатора для хостов в локальной сети. При совместном использовании IP-адреса и MAC-адреса два или более маршрутизатора могут работать как один виртуальный маршрутизатор.
Протокол резервирования обеспечивает механизм определения того, какой маршрутизатор должен играть активную роль в пересылке трафика. Он также определяет, когда роль пересылки должна взять на себя резервный маршрутизатор. Переход от одного маршрутизатора пересылки к другому прозрачен для конечных устройств.

HSRP — это запатентованный протокол Cisco FHRP, предназначенный для прозрачного аварийного переключения IP-устройства первого перехода.
HSRP обеспечивает высокую доступность сети, обеспечивая избыточность маршрутизации первого перехода для IP-хостов в сетях, настроенных с IP-адресом шлюза по умолчанию.
Преимущество HSRP
По умолчанию, после того как маршрутизатор становится активным маршрутизатором, он остается активным маршрутизатором, даже если другой маршрутизатор подключается к сети с более высоким приоритетом HSRP.
Чтобы принудительно запустить новый процесс выбора HSRP, когда маршрутизатор с более высоким приоритетом подключается к сети, необходимо включить вытеснение с помощью  резервной  команды интерфейса вытеснения. Преимущественное прерывание — это способность маршрутизатора HSRP инициировать процесс переизбрания. При включенном вытеснении маршрутизатор, подключенный к сети с более высоким приоритетом HSRP, возьмет на себя роль активного маршрутизатора.

\subsection{Какие роли исполняют активный, резервный и виртуальный маршрутизатор? Каким образом происходит процесс
выбора активного маршрутизатора?}

Активный маршрутизатор (Active Router) — маршрутизатор, который исполняет роль виртуального маршрутизатора, чтобы пересылать пакеты из одной подсети в другую;
Резервный маршрутизатор (Standby Router) — маршрутизатор, который используется как резервный виртуальный маршрутизатор, ожидающий отказа Active Router в пределах одной HSRP группы;

HRSP — это собственный FHRP Cisco, предназначенный для прозрачного аварийного переключения устройства IPv4 первого перехода. HSRP обеспечивает высокую доступность сети, обеспечивая избыточность маршрутизации первого перехода для хостов IPv4 в сетях, настроенных с адресом шлюза IPv4 по умолчанию. HSRP используется в группе маршрутизаторов для выбора активного устройства и резервного устройства. В группе интерфейсов устройств активное устройство — это устройство, используемое для маршрутизации пакетов; резервное устройство — это устройство, которое берет на себя управление, когда активное устройство выходит из строя или когда выполняются предварительно заданные условия. Функция резервного маршрутизатора HSRP заключается в отслеживании рабочего состояния группы HSRP и быстром принятии на себя ответственности за пересылку пакетов в случае отказа активного маршрутизатора.


\subsection{Что происходит в случае сбоя активного маршрутизатора? Что произойдет, если в сети появится маршрутизатор
с более высоким приоритетом?}

Маршрутизатор может быть либо активным маршрутизатором HSRP, ответственным за пересылку трафика для сегмента, либо пассивным маршрутизатором HSRP в режиме ожидания, готовым взять на себя активную роль в случае сбоя активного маршрутизатора. Когда интерфейс настроен с использованием HSRP или впервые активирован с существующей конфигурацией HSRP, маршрутизатор отправляет и получает приветственные пакеты HSRP, чтобы начать процесс определения того, какое состояние он примет в группе HSRP.

Приоритет HSRP можно использовать для определения активного маршрутизатора. Маршрутизатор с наивысшим приоритетом HSRP станет активным маршрутизатором. По умолчанию приоритет HSRP равен 100. Если приоритеты равны, маршрутизатор с наивысшим численным адресом IPv4 выбирается в качестве активного маршрутизатора.

\subsection{В каком случае сработает приоритетное вытеснение маршрутизатора? Опишите принцип работы сетевой атаке DDoS.}

Чтобы принудительно запустить новый процесс выбора HSRP, когда маршрутизатор с более высоким приоритетом подключается к сети, необходимо включить вытеснение с помощью резервной команды интерфейса вытеснения. Преимущественное прерывание — это способность маршрутизатора HSRP инициировать процесс переизбрания. При включенном вытеснении маршрутизатор, подключенный к сети с более высоким приоритетом HSRP, возьмет на себя роль активного маршрутизатора.
Вытеснение позволяет маршрутизатору стать активным только в том случае, если он имеет более высокий приоритет. Маршрутизатор, включенный для вытеснения, с таким же приоритетом, но с более высоким IPv4-адресом, не будет вытеснять активный маршрутизатор.

Распределенный отказ в обслуживании (DDoS) — это скоординированная атака со стороны многих устройств, называемых зомби, с намерением ухудшить или остановить публичный доступ к веб-сайту и ресурсам организации.


\subsection{Дайте характеристику компонентам AAA. Как будет вести себя коммутатор в результате успешной атаки на таблицу MAC?}

AAA обеспечивает основную структуру для настройки контроля доступа на сетевом устройстве.
AAA — это способ контролировать, кому разрешен доступ к сети (аутентификация), что они могут делать, находясь там
(авторизация), и проверять, какие действия они выполняли при доступе к сети (учет)

Все таблицы MAC-адресов имеют фиксированный размер, и, следовательно, у коммутатора могут закончиться ресурсы для хранения MAC-адресов. Атаки с лавинной рассылкой MAC-адресов используют это ограничение, бомбардируя коммутатор поддельными исходными MAC-адресами до тех пор, пока таблица MAC-адресов коммутатора не заполнится.
Когда это происходит, коммутатор обрабатывает кадр как неизвестный одноадресный и начинает рассылать весь входящий трафик со всех портов той же VLAN без обращения к таблице MAC-адресов. Это условие теперь позволяет субъекту угрозы перехватывать все кадры, отправляемые с одного хоста на другой в локальной сети LAN или локальной сети VLAN.


\subsection{Опишите принцип работы атаки с двойным тегированием. В чем заключается опасность ARP атак?}

Злоумышленник отправляет коммутатору кадр 802.1Q с двойной меткой. Внешний заголовок имеет тег VLAN субъекта угрозы, который совпадает с собственным VLAN магистрального порта. Для целей этого примера предположим, что это VLAN 10. Внутренний тег — это VLAN-жертва, в данном примере VLAN 20.
Кадр поступает на первый коммутатор, который просматривает первую 4-байтовую метку 802.1Q. Коммутатор видит, что кадр предназначен для VLAN 10, которая является собственной VLAN. Коммутатор пересылает пакет через все порты VLAN 10 после удаления тега VLAN 10. Кадр не перемаркирован, потому что он является частью собственной VLAN. На данный момент тег VLAN 20 все еще не поврежден и не проверен первым коммутатором.
Кадр поступает на второй коммутатор, который не знает, что он должен быть предназначен для VLAN 10. Собственный трафик VLAN не помечается отправляющим коммутатором, как указано в спецификации 802.1Q. Второй коммутатор смотрит только на внутренний тег 802.1Q, вставленный злоумышленником, и видит, что кадр предназначен для VLAN 20, целевой VLAN. Второй коммутатор отправляет кадр целевому объекту или рассылает его, в зависимости от того, существует ли существующая запись в таблице MAC-адресов для целевого объекта.

Проблема заключается в том, что злоумышленник может отправить коммутатору произвольное ARP-сообщение, содержащее поддельный MAC-адрес, и коммутатор соответствующим образом обновит свою таблицу MAC-адресов. Таким образом, любой хост может претендовать на роль владельца любой комбинации IP- и MAC-адресов по своему выбору. В типичной атаке субъект угрозы может отправить незапрашиваемые ответы ARP другим узлам в подсети с MAC-адресом субъекта угрозы и IP-адресом шлюза по умолчанию.


\subsection{В чем заключается потенциальная опасность использование протокола CDP? Как поступит маршрутизатор, если на нем не настроен маршрут по умолчанию и пакет должен быть перенаправлен в сеть назначения, которая не указана в его таблице маршрутизации?}

Информация CDP чрезвычайно полезна при устранении неполадок в сети. Например, CDP можно использовать для проверки подключения на уровнях 1 и 2. Если администратор не может пропинговать непосредственно подключенный интерфейс, но все же получает информацию CDP, то проблема, скорее всего, связана с конфигурацией уровня 3.
Однако информация, предоставляемая CDP, также может быть использована злоумышленником для обнаружения уязвимостей сетевой инфраструктуры.

Неверный шлюз по умолчанию. Мошеннический сервер предоставляет неверный шлюз или IP-адрес своего хоста для создания атаки «человек посередине». Это может остаться незамеченным, поскольку злоумышленник перехватывает поток данных через сеть.


\subsection{Какие данные могут быть получены с помощью протокола CDP? Каким образом можно провести атаку STP протокола?}

Информация CDP включает IP-адрес устройства, версию программного обеспечения IOS, платформу, возможности и собственный VLAN. Устройство, получающее сообщение CDP, обновляет свою базу данных CDP.

Чтобы провести атаку с манипулированием STP, атакующий хост рассылает в широковещательном режиме блоки данных протокола моста STP (BPDU), содержащие изменения конфигурации и топологии, которые вызывают повторные вычисления связующего дерева. BPDU, отправленные атакующим хостом, сообщают о более низком приоритете моста в попытке быть избранным в качестве корневого моста.
В случае успеха атакующий хост становится корневым мостом, как показано на рисунке, и теперь может захватывать различные кадры, которые в противном случае были бы недоступны.
Эта атака STP смягчается путем реализации BPDU Guard на всех портах доступа. BPDU Guard обсуждается более подробно позже в курсе.

\subsection{В чем заключается опасность DHCP-спуфинга? Опишите метод сетевой атаки VLAN Hopping}

DHCP-спуфинг-атака.
Атака подмены DHCP происходит, когда мошеннический DHCP-сервер подключается к сети и предоставляет ложные параметры конфигурации IP законным клиентам. Мошеннический сервер может предоставить различную вводящую в заблуждение информацию:
Неверный шлюз по умолчанию. Мошеннический сервер предоставляет неверный шлюз или IP-адрес своего хоста для создания атаки «человек посередине». Это может остаться незамеченным, поскольку злоумышленник перехватывает поток данных через сеть.
Неправильный DNS-сервер. Мошеннический сервер предоставляет неверный адрес DNS-сервера, указывающий пользователю на гнусный веб-сайт.
еправильный IP-адреc. Мошеннический сервер предоставляет неверный IP-адрес, что фактически создает DoS-атаку на DHCP-клиент.
Атака с переключением VLAN позволяет видеть трафик из одной VLAN в другой VLAN без помощи маршрутизатора. В обычной атаке с переключением VLAN злоумышленник настраивает хост так, чтобы он действовал как коммутатор, чтобы воспользоваться преимуществами функции автоматического транкового порта, включенной по умолчанию на большинстве портов коммутатора.
Злоумышленник настраивает хост для подмены сигналов 802.1Q и собственного протокола Cisco Dynamic Trunking Protocol (DTP) на соединительную линию с подключаемым коммутатором. В случае успеха коммутатор устанавливает транковую связь с хостом, как показано на рисунке. Теперь злоумышленник может получить доступ ко всем VLAN на коммутаторе. Злоумышленник может отправлять и получать трафик в любой VLAN, эффективно переключаясь между VLAN.

