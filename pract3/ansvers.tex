\section{Ответы на вопросы}

\subsection{Что такое маршрутизация между VLAN? Какие бывают методы маршрутизации между VLAN?}

Маршрутизация между VLAN — это процесс маршрутизации
трафика между сетями VLAN с использованием выделенного
маршрутизатора или многоуровневого коммутатора. (процесс пересылки
сетевого трафика из одной VLAN в другую с использованием
маршрутизации называют маршрутизацией между VLAN.)

Способы маршрутизации между VLAN:
- с использованием устаревшего метода.
- с использованием метода Router-on-a-stick
- маршрутизация на основе коммутатора 3-го уровня

\subsection{Опишите устаревший метод маршрутизации между сетями VLAN. В чем заключается преимущество
маршрутизации между VLAN с помощью коммутатора уровня 3?}

Устаревший метод маршрутизации между VLAN - использование
маршрутизаторов с несколькими физическими интерфейсами. Каждый
интерфейс должен был быть подключён к отдельной сети и настроен с
определённой подсетью. При таком устаревшем подходе маршрутизация
между VLAN выполняется путём подключения различных физических
интерфейсов маршрутизатора к разным физическим портам коммутатора.
Порты коммутатора, подключённые к маршрутизатору, переводятся в режим
доступа, а каждый физический интерфейс назначается отдельной VLAN.
Каждый интерфейс маршрутизатора может принимать трафик из VLAN,
связанной с интерфейсом коммутатора, к которому она подключена, и
трафик можно направлять в другие VLAN, подключённые к другим
интерфейсам.

Использование устройства 3-го уровня обеспечивает возможность
управления передачей трафика между сегментами сети, в том числе
сегментами, которые были созданы с помощью VLAN. Как правило, полоса
пропускания коммутаторов 3-го уровня позволяет передавать миллионы
пакетов в секунду (pps), в то время как стандартные маршрутизаторы
поддерживают скорость коммутации от 100 тысяч до 1 миллиона пакетов в
секунду.

\subsection{Дайте характеристику методу маршрутизации Router-on-a-Stick. В чем заключается недостаток
устаревшего метода маршрутизации между сетями VLAN?}

Метод «router-on-a-stick» — это такой тип конфигурации
маршрутизатора, при котором один физический интерфейс маршрутизирует
трафик между несколькими VLAN.Топология router-on-a-stick полагается на
внешний маршрутизатор с подынтерфейсами, подключёнными через
транковые каналы к коммутатору 2-го уровня.

Интерфейс маршрутизатора настраивается для работы в качестве
транкового канала и подключается к порту коммутатора, который настроен в
режиме транка. Маршрутизатор выполняет маршрутизацию между VLAN,
принимая на транковом интерфейсе трафик с меткой VLAN, поступающий от
смежного коммутатора, и затем с помощью подынтерфейсов маршрутизируя
его между VLAN. Затем уже смаршрутизированный трафик посылается с
этого же физического интерфейса с меткой VLAN, соответствующей VLAN
назначения.

Минус устаревшего метода маршрутизации: В топологии
используются параллельные каналы для создания транков между
коммутаторами в целях агрегации и резервирования каналов. Однако
резервные каналы усложняют топологию и без должного управления могут
привести к проблемам с подключением.

\subsection{Опишите алгоритм настройки маршрутизации между сетями VLAN методом Router-on-a-Stick. В чем
заключается недостаток метода маршрутизации Router-on-a-Stick?}

При использовании метода router-on-a-stick на каждом логическом
подынтерфейсе необходимо настроить соответствующие IP-адреса и
параметры VLAN. Необходимо настроить транк и инкапсуляцию на
маршрутизаторе и на соответствующем порту коммутатора.

Маршрутизация между VLAN с использованием метода router-on-a-
stick не масштабируется при работе более 50 сетей VLAN.

\subsection{Опишите алгоритм настройки маршрутизации между VLAN с помощью коммутатора уровня 3. Дайте
определение понятию “подынтерфейс” }

\begin{enumerate}
    \item Включите маршрутизацию на коммутаторе
    \begin{itemize}
        \item Войдите в интерфейс командной строки коммутатора.
        \item Введите команду "ip routing" для включения маршрутизации.
    \end{itemize}

    \item Создайте VLAN и назначьте порты
    \begin{itemize}
        \item Создайте VLAN с помощью команды "vlan [номер VLAN]".
        \item Назначьте порты VLAN с помощью команды "interface vlan
        [номер VLAN]".
    \end{itemize}

    \item Настройте IP-адреса и шлюзы для VLAN
    \begin{itemize}
        \item Назначьте IP-адрес и маску подсети каждому VLAN с помощью
        команды "interface vlan [номер VLAN]".
        \item Настройте шлюз по умолчанию для каждого VLAN с помощью
        команды "ip default-gateway [IP-адрес шлюза]".
    \end{itemize}

    \item Создайте статические маршруты
    \begin{itemize}
        \item Создайте статические маршруты для пересылки трафика между
        VLAN.
        \item Используйте команду "ip route [сеть назначения] [маска подсети]
        [IP-адрес следующего перехода] [интерфейс]".
    \end{itemize}

    \item Проверьте конфигурацию
    \begin{itemize}
        \item Проверьте конфигурацию с помощью команды "show ip route".
        \item Убедитесь, что маршруты созданы правильно и что трафик
        пересылается между VLAN.
    \end{itemize}
\end{enumerate}

Подынтерфейсы — это программные виртуальные интерфейсы,
связанные с одним физическим интерфейсом.


\subsection{Опишите алгоритм настройки маршрутизации на коммутаторе уровня 3.
В чем заключается недостаток использования многоуровневых коммутаторов для маршрутизации между VLAN?
Внедрение маршрутизации между виртуальными локальными сетями }

Алгоритм настройки маршрутизации на коммутаторе уровня 3:
Шаг 1. Настройте маршрутизируемый порт.
Шаг 2. Включите маршрутизацию.
Шаг 3. Настройте маршрутизацию.
Шаг 4. Проверка маршрутизации.
Шаг 5. Проверьте подключение.

Недостаток в том, что для каждой VLAN нужен отдельный интерфейс
маршрутизатора.

\subsection{Какие неполадки могут возникнуть при настройке маршрутизации между VLAN и как их исправить?
В каком режиме должен находиться порт коммутатора при подключении его к маршрутизатору для
маршрутизации между VLAN методом Router-on-a-Stick? }
Неполадки при настройке маршрутизации между VLAN и способы их
исправления:
\begin{itemize}
    \item Отсутствующие сети VLAN \par
    Решение:
    \begin{itemize}
        \item Создайте (или повторно создайте) VLAN, если она не
        существует.
        \item Убедитесь, что порт хоста назначен правильной VLAN.
    \end{itemize}

    \item Проблемы магистрального порта коммутатора \par
    Решение:
    \begin{itemize}
        \item Убедитесь, что магистральные соединения настроены правильно.
        \item Убедитесь, что порт является магистральным портом и включен.
    \end{itemize}

    \item Неполадки в работе порта коммутатора \par
    Решение:
    \begin{itemize}
        \item Назначьте порт соответствующей сети VLAN.
        \item Убедитесь, что порт является портом доступа и включен.
        \item Неправильно настроен узел в неправильной подсети.
    \end{itemize}

    \item Неполадки в настройках маршрутизатора\par
    Решение:
    \begin{itemize}
        \item IPv4-адрес подынтерфейса маршрутизатора настроен
        неправильно.
        \item Подынтерфейс маршрутизатора назначается с идентификатором
        VLAN.
        \item Порт коммутатора работает в режиме транка.
    \end{itemize}
    \item
\end{itemize}


\subsection{Какими возможностями обладает коммутатор уровня 3 по сравнению с коммутатором уровня 2?
Между какими устройствами необходимо настроить магистральный канал при использовании метода
Router-on-a-Stick? }

Корпоративные локальные сети используют коммутаторы уровня 3 для
обеспечения маршрутизации между VLAN. Коммутаторы уровня 3
используют аппаратную коммутацию для достижения более высоких
скоростей обработки пакетов, чем маршрутизаторы. Коммутаторы уровня 3
также обычно используются в корпоративных сетях стойка уровня
распределения.
Возможности коммутатора уровня 3 включают в себя возможность
выполнения следующих действий:

\begin{itemize}
    \item Маршрутизация от одной VLAN к другой с использованием
    нескольких коммутируемых виртуальных интерфейсов (SVI).
    \item Преобразовать порт коммутатора уровня 2 в интерфейс уровня 3
    (т.е. маршрутизируемый порт). Маршрутизируемый порт —
    простой интерфейс 3-го уровня, аналогичный физическому
    интерфейсу на маршрутизаторе Cisco IOS.
\end{itemize}

При использовании метода Router-on-a-Stick между маршрутизаторами
и коммутаторами необходимо настроить магистральный канал.