\section*{\LARGE Выполнение работы}
\addcontentsline{toc}{section}{Выполнение работы}

\section{Создание сети и настройка основных параметров устройства}

В первой части лабораторной работы вам предстоит создать топологию сети и настроить базовые
параметры для узлов ПК и коммутаторов.

\subsection{Создайте сеть согласно топологии}
Подключите устройства, как показано в топологии, и подсоедините необходимые кабели.

\subsection{Настройте базовые параметры для маршрутизатора.}
\begin{enumerate}[a]
    \item Подключитесь к маршрутизатору с помощью консоли и активируйте привилегированный режим
    EXEC\@.
    \begin{verbatim}
        enabel
    \end{verbatim}
    \item Войдите в режим конфигурации.
    \begin{verbatim}
        config t
    \end{verbatim}
    \item Назначьте маршрутизатору имя устройства
    \begin{verbatim}
        hostname R1_Shestakov
    \end{verbatim}
    \item Отключите поиск DNS, чтобы предотвратить попытки маршрутизатора неверно преобразовывать
    введенные команды таким образом, как будто они являются именами узлов.
    \begin{verbatim}
        no ip domain lookup
    \end{verbatim}
    \item Назначьте class в качестве зашифрованного пароля привилегированного режима EXEC\@.
    \begin{verbatim}
        enable secret class
    \end{verbatim}
    \item Назначьте cisco в качестве пароля консоли и включите вход в систему по паролю.
    \begin{verbatim}
        line console 0
        password cisco
        login
    \end{verbatim}
    \item Установите cisco в качестве пароля виртуального терминала и активируйте вход
    \begin{verbatim}
        line vty 0 4
        password cisco
        login
    \end{verbatim}
    \item Зашифруйте открытые пароли.
    \begin{verbatim}
        service password-encryption
    \end{verbatim}
    \item Создайте баннер с предупреждением о запрете несанкционированного доступа к устройству.
    \begin{verbatim}
        banner motd # You must be authorizeded! #
    \end{verbatim}
    \item Сохраните текущую конфигурацию в файл загрузочной конфигурации.
    \begin{verbatim}
        copy running-config startup-config
    \end{verbatim}
    \item Настройте на маршрутизаторе время.
    \begin{verbatim}
        clock set 19:00:00 23 Feb 2024
    \end{verbatim}
\end{enumerate}

\subsection{Настройте базовые параметры каждого коммутатора.}

\begin{enumerate}[a]
    \item Присвойте коммутатору имя устройства.
    \begin{verbatim}
        enable
        config t
        hostname S1
    \end{verbatim}

    \item Отключите поиск DNS, чтобы предотвратить попытки маршрутизатора неверно преобразовывать
    введенные команды таким образом, как будто они являются именами узлов.
    \begin{verbatim}
        no ip domain lookup
    \end{verbatim}

    \item Назначьте class в качестве зашифрованного пароля привилегированного режима EXEC\@.
    \begin{verbatim}
        enable secret class
    \end{verbatim}

    \item Назначьте cisco в качестве пароля консоли и включите вход в систему по паролю.
    \begin{verbatim}
        line console 0
        password cisco
        login
    \end{verbatim}

    \item Установите cisco в качестве пароля виртуального терминала и активируйте вход.
    \begin{verbatim}
        line vty 0 15
        password cisco
        login
    \end{verbatim}

    \item Зашифруйте открытые пароли.
    \begin{verbatim}
        service password-encryption
    \end{verbatim}

    \item Создайте баннер с предупреждением о запрете несанкционированного доступа к устройству.
    \begin{verbatim}
        banner motd # You must be authorizeded! #
    \end{verbatim}

    \item Настройте на коммутаторах время.
    \begin{verbatim}
        clock set 19:00:00 23 Feb 2024
    \end{verbatim}

    \item Сохранение текущей конфигурации в качестве начальной.
    \begin{verbatim}
        copy running-config startup-config
    \end{verbatim}
\end{enumerate}

\subsection{Настройте узлы ПК.}
Адреса ПК можно посмотреть в таблице адресации.