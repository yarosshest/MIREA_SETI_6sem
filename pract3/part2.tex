\section{Создание сетей VLAN и назначение портов коммутатора}
Во второй части вы создадите VLAN, как указано в таблице выше, на обоих коммутаторах.
Затем вы назначите VLAN соответствующему интерфейсу и проверите настройки конфигурации.
Выполните следующие задачи на каждом коммутаторе

\subsection{Создайте сети VLAN на коммутаторах.}
\begin{enumerate}[a]
    \item Создайте и назовите необходимые VLAN на каждом коммутаторе из таблицы выше.
    \begin{verbatim}
        vlan 37
        name Management
        vlan 47
        name Sales
        vlan 57
        name Operations
        vlan 999
        name Parking_Lot
        vlan 1000
        name Native
    \end{verbatim}

    \item Настройте интерфейс управления и шлюз по умолчанию на каждом коммутаторе, используя
    информацию об IP-адресе в таблице адресации.
    \begin{verbatim}
        interface vlan 37
        *S1* ip address 192.168.37.11 255.255.255.0
        *S2* ip address 192.168.37.12 255.255.255.0
        no shutdown
        exit
        ip default-gateway 192.168.37.1
    \end{verbatim}

    \item Назначьте все неиспользуемые порты коммутатора VLAN Parking\_Lot, настройте их для
    статического режима доступа и административно деактивируйте их.
    Примечание.
    Команда interface range полезна для выполнения этой задачи с минимальным
    количеством команд.
    \begin{verbatim}
        *S1* interface range f0/2 - 4 , f0/7 - 24 , g0/1 - 2
        *S2* interface range f0/2 - 17 , f0/19 - 24 , g0/1 - 2
        switchport mode access
        switchport access vlan 999
        shutdown
    \end{verbatim}
\end{enumerate}

\subsection{Назначьте сети VLAN соответствующим интерфейсам коммутатора}
\begin{enumerate}[a]
    \item Назначьте используемые порты соответствующей VLAN (указанной в таблице VLAN выше) и
    настройте их для режима статического доступа.
    \begin{verbatim}
        *S1* interface f0/6
        *S2* interface f0/18
        switchport mode access
        *S1* switchport access vlan 47
        *S2* switchport access vlan 57
    \end{verbatim}

    \item Убедитесь, что VLAN назначены на правильные интерфейсы.
    \begin{verbatim}
        show vlan brief
    \end{verbatim}
\end{enumerate}

\img{img/png5}{Настрройка S1 S2}