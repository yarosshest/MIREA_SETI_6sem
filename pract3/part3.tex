\section{Конфигурация магистрального канала стандарта 802.1Q между коммутаторами}
В части 3 вы вручную настроите интерфейс F0/1 как транковый канал.

\subsection{Вручную настройте магистральный интерфейс F0/1 на коммутаторах S1 и S2}
\begin{enumerate}[a]
    \item Настройте интерфейс F0/1 как транковый для обоих коммутаторов.
    \begin{verbatim}
        interface f0/1
        switchport mode trunk
    \end{verbatim}

    \item Установите native VLAN 1000 на обоих коммутаторах.
    \begin{verbatim}
        switchport trunk native vlan 1000
    \end{verbatim}

    \item Укажите, что VLAN X+10, X+20, X+30 и 1000 могут проходить по транковому каналу
    \begin{verbatim}
        switchport trunk allowed vlan 37,47,57,1000
    \end{verbatim}

    \item Проверьте транковые каналы, native VLAN и разрешенные VLAN через транковые каналы.
    \begin{verbatim}
        show interfaces trunk
    \end{verbatim}
\end{enumerate}

\subsection{Вручную настройте магистральный интерфейс F0/5 на коммутаторе S1.}
\begin{enumerate}[a]
    \item Настройте интерфейс S1 F0/5 с теми же параметрами транкового канала, что и F0/1.
    Это транковый канал до маршрутизатора.
    \begin{verbatim}
        **S1**
        interface f0/5
        switchport mode trunk
        switchport trunk native vlan 1000
        switchport trunk allowed vlan 37,47,57,1000
    \end{verbatim}

    \item Сохраните текущую конфигурацию в файл загрузочной конфигурации.
    \begin{verbatim}
        copy running-config startup-config
    \end{verbatim}

    \item Проверьте транковый канал.
    \begin{verbatim}
        show interfaces trunk
    \end{verbatim}

\end{enumerate}

Что произойдет, если G0/0/1 на R1\_ФАМИЛИЯ будет отключен?
-- Не будет отобразаться

