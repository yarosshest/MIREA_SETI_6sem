\section{Ответы на вопросы}

\subsection{Для чего используется резервирование в коммутируемых сетях уровня 2? Опишите назначение
протокола STP.}

Резервирование является важной частью иерархической модели,
предотвращающей перебои в оказании сетевых сервисов пользователям. Для
сетей с резервированием требуется добавление физических путей, но
необходимо также предусмотреть и логическое резервирование. Наличие
альтернативных физических каналов для передачи данных по сети позволяет
пользователям получить доступ к сетевым ресурсам даже в случае сбоя
одного из каналов. Тем не менее избыточные маршруты в коммутируемой
сети Ethernet могут привести к возникновению физических и логических
петель 2-го уровня.

Протокол связующего дерева (STP) - это сетевой протокол
предотвращения петель, который обеспечивает избыточность при создании
топологии уровня 2 без петель. IEEE 802.1D является исходным стандартом
IEEE MAC соединения мостов для STP.


\subsection{Опишите негативные последствия наличия петель коммутации.
Почему такие петли не могут возникнуть на уровне 3?}

Без включения STP петли уровня 2 могут формироваться, что приводит
к бесконечному циклу широковещательных, многоадресных и неизвестных
одноадресных кадров. Это может привести к разрушению сети в течение
очень короткого промежутка времени.
Неизвестные одноадресные кадры, отправленные в циклическую сеть,
могут приводить к появлению дублирующихся кадров, поступающих на
целевое устройство. Неизвестный одноадресный кадр с коммутатора
формируется, когда у коммутатора нет MAC-адреса назначения в таблице
MAC-адресов, и он должен переслать этот кадр со всех своих портов, за
исключением входного порта.

Петли коммутации обычно возникают на уровне 2 (канальный
уровень), когда коммутаторы в сети не используют алгоритмы блокирования
портов и пакеты данных начинают бесконечно циркулировать между
портами. На уровне 3 (сетевом уровне) используются протоколы
маршрутизации, которые обеспечивают оптимальное направление передачи
данных и предотвращают возникновение петель коммутации.


\subsection{Какие типы рассылок могут привести к возникновению
петель коммутации? Дайте определение понятию “широковещательный
шторм”.}

Обычные и широковещательные.
Петля коммутации- состояние в сети, при котором происходит
бесконечная пересылка фреймов между коммутаторами, подключенными в
один и тот же сегмент сети.

Широковещательный шторм в состоянии петли - когда компьютер PC-1
посылает фрейм с широковещательным адресом назначения. В такой
ситуации на все компьютеры сети будут бесконечно рассылаться копии
фрейма.

\subsection{Для чего был придуман алгоритм связующего дерева и в чем
его суть? Дайте определение понятию BPDU.}

Протоколсвязующегодерева(STP) - это сетевой протокол,
реализованный для предотвращения петель. Для сетей Ethernet
протоколсвязующегодерева(STP) создает логическую архитектуру,
лишенную петель. Основная цельSTP- остановить петли моста и
вызываемое ими широковещательное излучение.
BPDU (Bridge Protocol Data Unit) — фрейм (единица данных)
протокола управления сетевыми мостами (IEEE 802.1d).

\subsection{Какие 4 этапа проходит протокол STP при построении
топологии без петель коммутации? Какие поля содержит BID?}

\begin{enumerate}
    \item Выбор «корневого» (root) коммутатора.
    \item Выбор «корневого» (root) порта.
    \item Назначение «выделенного» (designated) порта.
    \item Блокировка остальных портов в рамках алгоритма STP.
\end{enumerate}

BID - идентификатор моста (значение приоритета + расширенный
системный идентификатор (VLAN) + MAC-адрес коммутатора).

ИзначальноBridge IDсостоял из двух полей:
Приоритет— поле, которое позволяет административно влиять
на выборы корневого коммутатора. Размер — 2 байта,
MAC-адрес— используется как уникальный идентификатор,
который, в случае совпадения значений приоритетов, позволяет выбрать
корневой коммутатор. Так как MAC-адреса уникальны, то и Bridge ID
уникален, так что какой-то коммутатор обязательно станет корневым.


\subsection{Что представляет из себя значение поля “приоритет моста”?
Какое поле в BID будет учитываться при выборе корневого моста, если
приоритет моста у всех коммутаторов одинаковый?}

Приоритет— поле, которое позволяет административно влиять на
выборы корневого коммутатора. Размер — 2 байта.
MAC-адрес— используется как уникальный идентификатор, который,
в случае совпадения значений приоритетов, позволяет выбрать корневой
коммутатор. Так как MAC-адреса уникальны, то и Bridge ID уникален, так
что какой-то коммутатор обязательно станет корневым.

\subsection{Какое значение приоритета моста является наиболее
приоритетным и каков шаг для значений данного поля? Дайте
определение понятию “стоимость корневого пути”.}

Чтобы задать корневой мост, для BID выбранного коммутатора
настраивается минимальный приоритет. Для настройки приоритета моста
используется команда bridge priority. Значение приоритета может находиться
в диапазоне от 0 до 65 535, но шаг между значениями составляет 4 096.
Значение по умолчанию — 32 768.
Стоимость маршрута до корневого свича (Root Path Cost)— это
поле в BPDU, которое используется для определения стоимости маршрута от
каждого порта коммутатора до корневого свича.

\subsection{Что представляет из себя значение поля “расширенный идентификатор
системы”? Для чего данное поле было добавлено в BID?}

Расширенный идентификатор системы используется для определения
экземпляра протокола STP

\img{img/png8}{BID}

\subsection{Каким образом происходит выбор корневого порта? Какие критерии
использует коммутатор для выбора роли порта при наличии
нескольких путей равной стоимости к корневому мосту?}

\begin{itemize}
    \item Сначала сравниваются значения стоимости пути к корневому мосту,
    порт сменьшей стоимостьюпути выбирается корневым;
    \item Если имеются 2 порта с одинаковой минимальной стоимостью пути
    к корневому мосту и порты подключены к разным соседним
    коммутаторам, то каждый порт смотрит в пришедший кадр BPDU и
    находит там значение BID (то есть BridgeID соседнего коммутатора, к
    которому подключен этот порт). Порт, для которого BIDменьше,
    становится корневым;
    \item Если BridgeID равны или порты подключены к одному и тому же
    соседнему коммутатору, то каждый порт смотрит в пришедший кадр
    BPDU и находит там значение PortID (это PortID порта соседнего
    коммутатора, куда подключён данный порт). Порт, для которого
    данное значениеменьше, становится корневым.
\end{itemize}

\subsection{Каким образом происходит выбор назначенного порта? Какие
состояния портов используются в протоколе STP?}

Порт коммутатора, который имеет кратчайший путь к корневому
коммутатору - называется «назначенным».

\begin{itemize}
    \item Каждый сегмент (путь) имеет свой назначенный порт.
    \item Назначенные порты определяются на всех коммутаторах
    (корневых и нет).
\end{itemize}

Если два порта имеют одинаковую стоимость, сначала учитывается
идентификатор устройства (Bridge ID), а затем идентификатор порта (Port
ID).
Все остальные порты переходят в альтернативный статус и блокируются.

\subsection{В чем особенность протокола PVST? Дайте краткую характеристику
протоколу RSTP.}

Протокол PVST (Per-VLAN Spanning Tree Protocol) обеспечивает
построение и использование на коммутаторе независимых процессов
протокола STP (Spanning Tree Protocol) для каждой из
сконфигурированных на нем виртуальных сетей. Каждый из
магистральных портов такого коммутатора принимает, обрабатывает и
формирует стандартные конфигурационные сообщения BPDU-C отдельно
для каждой из VLAN, трафик которых через него передается. В результате
обмена этими сообщениями порт может перейти в активное состояние для
некоторых из VLAN и остаться в заблокированном состоянии для трафика
других виртуальных сетей. Поскольку протокол PVST предполагает
использование внешней схемы ISL для маркирования кадров, то для его
реализации в каждой из VLAN классический протокол STP может быть
использован в PVST без каких-либо изменений алгоритма. При этом,
естественно, остаются неизменными форматы и алгоритмы формирования
всех сообщений протокола STP. Основным недостатком протокола PVST
является невозможность его применения в тех локальных сетях, где
используются различные схемы кодирования принадлежности кадров к
VLAN (ISL и IEEE 802.1Q).

Протокол RSTP (Rapid Spanning Tree Protocol) является улучшенной
версией протокола Spanning Tree Protocol (STP), предназначенного для
обеспечения избыточности и безопасности в сетях Ethernet. Вот краткая
характеристика RSTP:

\begin{itemize}
    \item Быстрое восстановление: RSTP обеспечивает более быстрое
    сходимое время, чем его предшественник STP, благодаря
    уменьшению времени протоколирования портов и быстрому
    переводу портов в активное состояние.
    \item Поддержка портов: RSTP включает в себя три основных типа
    портов: корневые порты (Root Port), порты дизайнации (Designated
    Port) и альтернативные порты (Alternate Port). Это позволяет
    управлять трафиком в сети и обеспечивать избыточность
    соединений.
    \item Переходные состояния: RSTP уменьшает время, требуемое для
    переходных состояний, что позволяет сети быстрее
    восстанавливаться после изменений в топологии сети.
    \item Обратная совместимость: RSTP совместим с протоколом STP, что
    позволяет постепенно обновлять существующие сети без
    необходимости полной переконфигурации.
    \item Избыточность: Протокол RSTP поддерживает избыточные
    соединения, позволяя настройке резервных путей для случаев отказа
    основных соединений.
    \item Улучшенная безопасность: RSTP включает в себя механизмы
    защиты от циклов и петель, что повышает безопасность и
    стабильность сети.
\end{itemize}

В целом, RSTP является эффективным протоколом для управления
топологией сети Ethernet, обеспечивая быстрое восстановление и
избыточность в случае сбоев или изменений в сетевой структуре.

\subsection{Охарактеризуйте состояния, в которых может находиться порт при
использовании протокола RSTP. Для чего нужно использовать
функцию PortFast и для каких портов коммутатора?}

Порты в протоколе RSTP могут находиться в различных состояниях в
зависимости от их роли и текущей сетевой топологии. Вот основные
состояния портов в RSTP:

\begin{itemize}
    \item Discarding (Отбрасывание): Порт находится в этом состоянии, когда
    он не участвует в пересылке данных. Он прослушивает BPDU
    (Bridge Protocol Data Unit) для определения топологии сети и может
    перейти в состояние forwarding или blocking в зависимости от
    информации, полученной от других коммутаторов.
    \item Learning (Обучение): Порт находится в этом состоянии, когда он
    начинает изучать MAC-адреса, проходящие через него. В этом
    состоянии порт продолжает прослушивать трафик, но еще не
    пересылает его.
    \item Forwarding (Пересылка): Порт находится в активном состоянии и
    пересылает трафик между сегментами сети.
\end{itemize}

Функция PortFast используется для ускорения включения портов в
состояние forwarding, обеспечивая моментальное переход из состояния
отключения (disabled) в состояние forwarding при включении порта. Это
полезно для портов, подключенных к устройствам конечных
пользователей, таким как компьютеры или IP-телефоны, где отключение
порта на короткое время может вызвать временные проблемы с
подключением к сети. Обычно функция PortFast применяется к портам, не
подключенным к другим коммутаторам, чтобы избежать возможности
образования петель в сети при быстром переходе порта в состояние
forwarding.


\subsection{Для чего необходимо использовать функцию BPDU guard и для каких
портов коммутатора? Какое решение можно использовать в качестве
альтернативы протоколу STP?}

Функция BPDU guard необходима для защиты сети от возможных
проблем, связанных с появлением нежелательных устройств, таких как
коммутаторы или маршрутизаторы, подключенные к портам,
настроенным как хостовые порты с помощью функции PortFast. Эта
функция следит за входящими BPDU на таких портах и при их
обнаружении мгновенно отключает порт, чтобы предотвратить возможное
образование петель в сети.

BPDU guard обычно применяется к портам, которые должны быть
хостовыми портами (например, портам, к которым подключаются
компьютеры, принтеры или IP-телефоны), чтобы предотвратить
подключение коммутаторов или других устройств, которые могут создать
петлю в сети.

В качестве альтернативы протоколу STP (Spanning Tree Protocol) можно
использовать протоколы, разработанные для устранения недостатков STP
и обеспечения более эффективного использования сетевых ресурсов.
Некоторые из таких альтернатив включают в себя:

\begin{itemize}
    \item Rapid Spanning Tree Protocol (RSTP): Улучшенная версия протокола
    STP, которая обеспечивает более быстрое восстановление сети и
    меньшее время сходимости.
    \item Multiple Spanning Tree Protocol (MSTP): Позволяет группировать
    виртуальные LAN (VLAN) в единый экземпляр протокола Spanning
    Tree, что позволяет эффективно использовать пропускную
    способность сети.
    \item Multiple Spanning Tree Protocol (MSTP): Позволяет группировать
    виртуальные LAN (VLAN) в единый экземпляр протокола Spanning
    Tree, что позволяет эффективно использовать пропускную
    способность сети.
    \item Shortest Path Bridging (SPB): Еще один протокол, разработанный для
    замены STP и обеспечения более быстрой и эффективной
    сходимости сети. Он основан на механизмах маршрутизации и
    технологии мультиплексирования.
\end{itemize}

Эти альтернативы обеспечивают более высокую производительность и
гибкость по сравнению с традиционным STP, что делает их
привлекательным выбором для современных сетей.