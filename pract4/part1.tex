\section*{\LARGE Выполнение работы}
\addcontentsline{toc}{section}{Выполнение работы}


\section{Создание сети и настройка основных параметровустройства}
В части 1 вам предстоит настроить топологию сети и основные параметры маршрутизаторов.

\subsection{Создайте сеть согласно топологии.}
Подключите устройства, как показано в топологии, и подсоедините необходимые кабели.

\subsection{Выполните инициализацию и перезагрузку коммутаторов.}
\begin{verbatim}
    enable
    erase startup-config
    reload
\end{verbatim}

\subsection{Настройте базовые параметры каждого коммутатора.}
\begin{enumerate}[a]
    \item Отключите поиск DNS\@.
    \begin{verbatim}
        enabel
        config t
        no ip domain lookup
    \end{verbatim}

    \item Присвойте имена устройствам в соответствии с топологией.
    \begin{verbatim}
        hostname S1_Shestakov
    \end{verbatim}

    \item Назначьте class в качестве зашифрованного пароля доступа к привилегированному режиму.
    \begin{verbatim}
        enable secret class
        service password-encryption
    \end{verbatim}

    \item Назначьте cisco в качестве паролей консоли и VTY и активируйте вход для консоли и VTY каналов.
    \begin{verbatim}
        line console 0
        password cisco
        login

        line vty 0 15
        password cisco
        login
    \end{verbatim}

    \item Настройте logging synchronous для консольного канала
    \begin{verbatim}
        logging synchronous
    \end{verbatim}

    \item Настройте баннерное сообщение дня (MOTD) для предупреждения пользователей о запрете
    несанкционированного доступа.
    \begin{verbatim}
        banner motd # You must be authorizeded! #
    \end{verbatim}

    \item Задайте IP-адрес, указанный в таблице адресации для VLAN 1 на обоих коммутаторах.
    \begin{verbatim}
        interface vlan 1
        *S1* ip address 192.168.28.1 255.255.255.0
        *S2* ip address 192.168.28.2 255.255.255.0
        *S3* ip address 192.168.28.3 255.255.255.0
        no shutdown
    \end{verbatim}

    \item Скопируйте текущую конфигурацию в файл загрузочной конфигурации.
    \begin{verbatim}
        copy running-config startup-config
    \end{verbatim}
\end{enumerate}

\img{img/png1}{Настройка коммутаторов}

\subsection{Проверьте связь.}
Проверьте способность компьютеров обмениваться эхо-запросами.
Успешно ли выполняется эхо-запрос от коммутатора S1\_ФАМИЛИЯ на
коммутатор S2? -- да
Успешно ли выполняется эхо-запрос от коммутатора
S1\_ФАМИЛИЯ на коммутатор S3? -- да
Успешно ли выполняется эхо-запрос от
коммутатора S2 на коммутатор S3? -- да
Выполняйте отладку до тех пор, пока ответы на все вопросы не будут положительными

\img{img/png2}{Ping коммутаторов}
