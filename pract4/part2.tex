\section{Определение корневого моста}
Для каждого экземпляра протокола spanning-tree (коммутируемая сеть LAN или широковещательный
домен) существует коммутатор, выделенный в качестве корневого моста.
Корневой мост служит точкой привязки для всех расчётов протокола spanning-tree, позволяя определить избыточные пути,
которые следует заблокировать.
Процесс выбора определяет, какой из коммутаторов станет корневым мостом.
Коммутатор с наименьшим значением идентификатора моста (BID) становится корневым мостом.
Идентификатор BID состоит из значения приоритета моста, расширенного идентификатора системы и MAC-адреса коммутатора.
Значение приоритета может находиться в диапазоне от 0 до 65535 с шагом 4096.
По умолчанию используется значение 32768.


\subsection{Отключите все порты на коммутаторах.}
\begin{verbatim}
    interface range f0/1-24, g0/1-2
    shutdown
\end{verbatim}

\subsection{Настройте подключенные порты в качестве транковых.}
\begin{verbatim}
    interface range f0/1-4
    switchport mode trunk
\end{verbatim}

\subsection{Включите порты F0/2 и F0/4 на всех коммутаторах.}
\begin{verbatim}
    interface range f0/2, f0/4
    no shutdown
\end{verbatim}

\subsection{Отобразите данные протокола spanning-tree.}
Введите команду show spanning-tree на всех трех коммутаторах.
Приоритет идентификатора моста рассчитывается путем сложения значений приоритета и расширенного идентификатора системы.
Расширенным идентификатором системы всегда является номер сети VLAN\@.

\begin{enumerate}[a]
    \item Создайте и назовите необходимые VLAN на каждом коммутаторе из таблицы выше.
    \begin{verbatim}
        vlan 37
        name Management
        vlan 47
        name Sales
        vlan 57
        name Operations
        vlan 999
        name Parking_Lot
        vlan 1000
        name Native
    \end{verbatim}

    \item Настройте интерфейс управления и шлюз по умолчанию на каждом коммутаторе, используя
    информацию об IP-адресе в таблице адресации.
    \begin{verbatim}
        interface vlan 37
        *S1* ip address 192.168.37.11 255.255.255.0
        *S2* ip address 192.168.37.12 255.255.255.0
        no shutdown
        exit
        ip default-gateway 192.168.37.1
    \end{verbatim}

    \item Назначьте все неиспользуемые порты коммутатора VLAN Parking\_Lot, настройте их для
    статического режима доступа и административно деактивируйте их.
    Примечание.
    Команда interface range полезна для выполнения этой задачи с минимальным
    количеством команд.
    \begin{verbatim}
        interface vlan 37
        *S1* ip address 192.168.37.11 255.255.255.0
        *S2* ip address 192.168.37.12 255.255.255.0
        no shutdown
        exit
        ip default-gateway 192.168.37.1
    \end{verbatim}
\end{enumerate}

\subsection{Назначьте сети VLAN соответствующим интерфейсам коммутатора}
\begin{enumerate}[a]
    \item Назначьте используемые порты соответствующей VLAN (указанной в таблице VLAN выше) и
    настройте их для режима статического доступа.
    \begin{verbatim}
        *S1* interface f0/6
        *S2* interface f0/18
        switchport mode access
        *S1* switchport access vlan 47
        *S2* switchport access vlan 57
    \end{verbatim}

    \item Убедитесь, что VLAN назначены на правильные интерфейсы.
    \begin{verbatim}
        show vlan brief
    \end{verbatim}
\end{enumerate}