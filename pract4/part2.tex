\section{Определение корневого моста}
Для каждого экземпляра протокола spanning-tree (коммутируемая сеть LAN или широковещательный
домен) существует коммутатор, выделенный в качестве корневого моста.
Корневой мост служит точкой привязки для всех расчётов протокола spanning-tree, позволяя определить избыточные пути,
которые следует заблокировать.
Процесс выбора определяет, какой из коммутаторов станет корневым мостом.
Коммутатор с наименьшим значением идентификатора моста (BID) становится корневым мостом.
Идентификатор BID состоит из значения приоритета моста, расширенного идентификатора системы и MAC-адреса коммутатора.
Значение приоритета может находиться в диапазоне от 0 до 65535 с шагом 4096.
По умолчанию используется значение 32768.


\subsection{Отключите все порты на коммутаторах.}
\begin{verbatim}
    interface range f0/1-24, g0/1-2
    shutdown
\end{verbatim}

\subsection{Настройте подключенные порты в качестве транковых.}
\begin{verbatim}
    interface range f0/1-4
    switchport mode trunk
\end{verbatim}

\subsection{Включите порты F0/2 и F0/4 на всех коммутаторах.}
\begin{verbatim}
    interface range f0/2, f0/4
    no shutdown
\end{verbatim}

\subsection{Отобразите данные протокола spanning-tree.}
Введите команду show spanning-tree на всех трех коммутаторах.
Приоритет идентификатора моста рассчитывается путем сложения значений приоритета и расширенного идентификатора системы.
Расширенным идентификатором системы всегда является номер сети VLAN\@.

\begin{verbatim}
    show spanning-tree
\end{verbatim}

Какой коммутатор является корневым мостом?
Почему этот коммутатор был выбран протоколом spanning-tree в качестве корневого моста?
Какой порт отображается в качестве альтернативного и в настоящее время заблокирован?
Почему протокол spanning-tree выбрал этот порт в качестве невыделенного (заблокированного) порта?
