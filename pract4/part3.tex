\section{Наблюдение за процессом выбора протоколом STPпорта,исходя из стоимости портов}
Алгоритм протокола spanning-tree (STA) использует корневой мост как точку привязки, после чего
определяет, какие порты будут заблокированы, исходя из стоимости пути.
Порт с более низкой стоимостью пути является предпочтительным.
Если стоимости портов равны, процесс сравнивает BID\@.
Если BID равны, для определения корневого моста используются приоритеты портов.
Наиболее низкие значения являются предпочтительными.
В части 3 вам предстоит изменить стоимость порта, чтобы определить, какой порт будет заблокирован протоколом
spanning-tree.


\subsection{Определите коммутатор с заблокированным портом.}
При текущей конфигурации только один коммутатор может содержать заблокированный протоколом
STP порт.
Выполните команду show spanning-tree на обоих коммутаторах некорневого моста и посмотрите, какой порт был заблокирован.

\begin{verbatim}
    *S3* show spanning-tree
    *S2* show spanning-tree
\end{verbatim}

\img{img/png4}{spanning-tree S2 -- S3}

\subsection{Измените стоимость порта.}
Помимо заблокированного порта, единственным активным портом на этом коммутаторе является порт,
выделенный в качестве порта корневого моста.
Уменьшите стоимость этого порта корневого моста до 18, выполнив команду spanning-tree cost 18 режима конфигурации
интерфейса.


\begin{verbatim}
    *S2*

    config t
    interface f0/2
    spanning-tree cost 18
\end{verbatim}

\subsection{Просмотрите изменения протокола spanning-tree}
Повторно выполните команду show spanning-tree на обоих коммутаторах некорневого моста.
Обратите внимание, что ранее заблокированный порт теперь является назначенным портом, и
протокол spanning-tree теперь блокирует порт на другом коммутаторе некорневого моста.
Почему протокол spanning-tree заменяет ранее заблокированный порт на назначенный порт
и блокирует порт, который был назначенным портом на другом коммутаторе?
\begin{verbatim}
    *S2* show spanning-tree
    *S3* show spanning-tree

    STP сначала проверяет стоимость пути.
    Порт с более низкой стоимостью пути всегда
    будет предпочтительнее порта с более высокой стоимостью пути.
\end{verbatim}

\img{img/png6}{spanning-tree коммутаторов}

\subsection{Удалите изменения стоимости порта.}
\begin{enumerate}[a]
    \item Выполните команду no spanning-tree cost 18 режима конфигурации интерфейса, чтобы удалить
    запись стоимости, созданную ранее.
    \begin{verbatim}
        no spanning-tree cost 18
    \end{verbatim}

    \item Повторно выполните команду show spanning-tree, чтобы подтвердить, что протокол STP сбросил
    порт на коммутаторе некорневого моста, вернув исходные настройки порта.
    Протоколу STP требуется примерно 30 секунд, чтобы завершить процесс перевода порта.
\end{enumerate}

\img{img/png7}{spanning-tree коммутаторов}