\section{Наблюдение за процессом выбора протоколом STPпорта,исходя из стоимости портов}
Если стоимости портов равны, процесс сравнивает BID. Если BID равны, для определения корневого
моста используются приоритеты портов.
Значение приоритета по умолчанию — 128.
STP объединяет приоритет порта с номером порта, чтобы разорвать связи.
Наиболее низкие значения являются предпочтительными.
В части 4 вам предстоит активировать избыточные пути до каждого из коммутаторов, чтобы просмотреть, каким образом
протокол STP выбирает порт с учетом приоритета портов.

\begin{enumerate}[a]
    \item Включите порты F0/1 и F0/3 на всех коммутаторах.
    \begin{verbatim}
        config t
        interface range f0/1, f0/3
        no shutdown
        end
    \end{verbatim}

    \item Подождите 30 секунд, чтобы протокол STP завершил процесс перевода порта, после чего
    выполните команду show spanning-tree на коммутаторах некорневого моста.
    Обратите внимание, что порт корневого моста переместился на порт с меньшим номером, связанный с коммутатором
    корневого моста, и заблокировал предыдущий порт корневого моста.

\end{enumerate}

Какой порт выбран протоколом STP в качестве порта корневого моста на каждом коммутаторе некорневого моста?**
\begin{verbatim}
    S2: f0/1
    S3: f0/3
\end{verbatim}
Почему протокол STP выбрал эти порты в качестве портов корневого моста на этих коммутаторах?**

Он взял с наибольшим значением порта 128.3;
Cледовательно, STP использовал номер порта, чтобы разорвать связь.
Он выбрал порт с большим номером в качестве корневого порта
и заблокировал порт с меньшим номером,
указав избыточный путь к корневому мосту.
