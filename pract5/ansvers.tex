\section{Ответы на вопросы}

\subsection{Дайте определение понятию “агрегирование каналов”. Опишите
преимущества технологии EtherChannel.}

Агрегирование каналов — это объединение нескольких физических
портов в одну логическую магистраль на канальном уровне модели OSI с
целью образования высокоскоростного канала передачи данных и
повышения отказоустойчивости. Все избыточные связи в одном
агрегированном канале остаются в рабочем состоянии, а имеющийся трафик
распределяется между ними для достижения балансировки нагрузки.

Технология EtherChannel имеет много достоинств:

\begin{itemize}
    \item Большинство задач конфигурации выполняется на интерфейсе
    EtherChannel, а не на отдельных портах.
    \item EtherChannel использует существующие порты коммутатора.
    \item Между каналами, которые являются частью одного и того же
    EtherChannel, происходит распределение нагрузки. Может быть
    реализован один или несколько методов балансировки нагрузки.
    \item EtherChannel создает объединение, которое рассматривается, как один
    логический канал. Если между двумя коммутаторами существует
    несколько объединений EtherChannel, протокол STP может
    блокировать одно из объединений во избежание петель коммутации.
    \item EtherChannel предоставляет функции избыточности, поскольку общий
    канал считается одним логическим соединением. Кроме того, потеря
    одного физического соединения в пределах канала не приводит к
    изменению в топологии. Поэтому перерасчет STP не требуется.
\end{itemize}

\subsection{Опишите назначение технологии EtherChannel. Какие
ограничения существуют при использовании технологии
EtherChannel?}

EtherChannel — это технология агрегации каналов, которая
группирует несколько физических каналов Ethernet вместе в один логический
канал. Он используется для обеспечения отказоустойчивости, распределения
нагрузки, увеличения пропускной способности и избыточности между
коммутаторами, маршрутизаторами и серверами.

EtherChannel имеет следующие ограничения реализации:
\begin{itemize}
    \item Нельзя одновременно использовать разные типы интерфейсов.
    \item Все каналы EtherChannel могут содержать до восьми совместимо
    настроенных Ethernet-портов.
    \item Коммутатор Cisco Catalyst 2960 уровня 2 в настоящее время
    поддерживает до шести каналов EtherChannel.
    \item Конфигурация порта отдельного участника группы EtherChannel
    должна выполняться согласованно на обоих устройствах.
    \item Каждый канал EtherChannel имеет логический интерфейс
    агрегированного канала. Настройка интерфейса агрегированного
    канала применяется на все физические интерфейсы, связанные с этим
    каналом.
\end{itemize}

\subsection{Дайте характеристику протоколу PAgP. Какие настройки
должны иметь все порты в группе для удачного создания
агрегированного канала?}

PAgP — это проприетарный протокол Cisco, который предназначен
для автоматизации создания каналов EtherChannel. Когда канал EtherChannel
настраивается с помощью PAgP, пакеты PAgP пересылаются между портами
с поддержкой EtherChannel в целях согласования создания канала. Когда
PAgP определяет совпадающие соединения Ethernet, он группирует их в
канал EtherChannel. Далее EtherChannel добавляется в дерево кратчайших
путей как один порт.

В EtherChannel все порты обязательно должны иметь одинаковую
скорость, одинаковые настройки дуплекса и одинаковые настройки VLAN.

\subsection{Перечислите и охарактеризуйте режимы работы протокола
PAgP. При настройке каких режимов PAgP на обоих концах
будет невозможно создать агрегированный канал (перечислите 2
сценария)?}

\begin{itemize}
    \item On (Вкл) — этот режим принудительно назначает интерфейс в
    канал без использования PAgP. Интерфейсы, настроенные в режиме
    On (Вкл), не обмениваются пакетами PAgP.
    \item PAgP desirable (рекомендуемый) — этот режим PAgP помещает
    интерфейс в активное состояние согласования, в котором
    интерфейс инициирует согласование с другими интерфейсами
    путем отправки пакетов PAgP.
    \item PAgP auto (автоматический) — этот режим PAgP помещает
    интерфейс в пассивное состояние согласования, в котором
    интерфейс отвечает на полученные пакеты PAgP, но не инициирует
    согласование PAgP.
\end{itemize}

2 сценария, когда на обоих концах будет невозможно создать
агрегированный канал:

\begin{enumerate}
    \item PAgP auto на обоих концах.
    \item On на одном конце и PAgP desirable или PAgP auto на другом
    конце.
\end{enumerate}

\subsection{Дайте характеристику протоколу LACP. Перечислите и
охарактеризуйте режимы работы протокола LACP.}

LACP определяется стандартом IEEE (802.3ad), который обеспечивает
возможность объединения нескольких физических портов для создания
единого логического канала. LACP обеспечивает возможность согласования
коммутатором автоматического объединения путем отправки пакетов LACP
на другой коммутатор. Протокол LACP можно использовать для упрощения
работы с каналами EtherChannel в неоднородных средах.

Режимы LACP:
\begin{itemize}
    \item On (Вкл) — этот режим принудительно помещает интерфейс в
    канал без использования LACP. Интерфейсы, настроенные в
    режиме On, не обмениваются пакетами LACP.
    \item LACP active (активный) — в этом режиме LACP порт
    помещается в активное состояние согласования. В этом состоянии
    порт инициирует согласование с другими портами путем отправки
    пакетов LACP.
    \item LACP passive (пассивный) — в этом режиме LACP порт
    помещается в пассивное состояние согласования. В этом состоянии
    порт отвечает на полученные пакеты LACP, но не инициирует
    согласование пакетов LACP.
\end{itemize}

\subsection{При настройке каких режимов LACP на обоих концах будет
невозможно создать агрегированный канал (перечислите 2
сценария)? Опишите алгоритм создания агрегированного
канала на коммутаторе.}

Два сценария, когда на обоих концах будет невозможно создать
агрегированный канал:

\begin{enumerate}
    \item On на одном конце и LACP active или LACP passive на другом
    конце.
    \item LACP active на одном конце и LACP passive на другом конце.
\end{enumerate}

Алгоритм создания агрегированного канала на коммутаторе:

\begin{enumerate}
    \item Подключитесь к коммутатору.
    \item Настройте каждый физический порт, который вы хотите
    объединить в порт-транк, удостоверившись, что порты находятся в
    одной и той же VLAN и обладают совместимыми параметрами.
    \item Создайте логический интерфейс порт-транка (или
    агрегированный интерфейс) на коммутаторе и добавьте в него
    указанные физические порты.
    \item Настройте параметры порт-транка.
    \item Проверьте правильность настройки, убедившись, что порт-
    транк создан и физические порты правильно агрегированы.
\end{enumerate}


\subsection{Опишите взаимодействие протокола STP с технологией
EtherChannel. Какие два метода балансировки нагрузки могут
быть реализованы с технологией EtherChannel}

Когда протокол STP взаимодействует с технологией EtherChannel, он
учитывает логический канал, созданный с помощью EtherChannel, как одно
соединение. Это позволяет балансировать нагрузку на портах EtherChannel и
обеспечивает надежность работы сети, так как STP будет рассматривать все
физические порты EtherChannel как одно соединение и будет отключать
избыточные пути для предотвращения возникновения петель.

Два метода балансировки нагрузки:

\begin{enumerate}
    \item Балансировка нагрузки по MAC-адресам или IP-адресам.
    \item Балансировка нагрузки по портам.
\end{enumerate}

\subsection{Какие параметры обязательно должны быть одинаковыми на
всех интерфейсах EtherChannel для его корректного
функционирования? Перечислите распространенные проблемы,
    с которыми можно столкнуться при работе с EtherChannel.}

В EtherChannel все порты обязательно должны иметь одинаковую
скорость, одинаковые настройки дуплекса и одинаковые настройки VLAN.

Распространенные проблемы EtherChannel:

\begin{itemize}
    \item Назначенные порты в EtherChannel не являются частью одной
    VLAN или не настроены как транки. Порты с различными native
    VLAN не могут образовать EtherChannel.
    \item Транк был настроен на некоторых портах, которые составляют
    EtherChannel, но не на всех из них. Не рекомендуется настраивать
    режим транкинга на отдельных портах, составляющих EtherChannel.
    При настройке магистрального канала в EtherChannel проверьте
    режим транкинга в EtherChannel.
    \item Если допустимый диапазон VLAN не совпадает, порты не
    формируют EtherChannel, даже если PAgP установлен в режим auto
    или desirable.
    \item Параметры динамического согласования для PAgP и LACP не
    совместимы на обоих концах EtherChannel.
\end{itemize}