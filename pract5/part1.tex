\section*{\LARGE Выполнение работы}
\addcontentsline{toc}{section}{Выполнение работы}


\section{Настройка основных параметров коммутатора}
В части 1 вы настроите топологию сети и такие базовые параметры, как IP-адреса интерфейсов,
доступ к устройствам и пароли.

\subsection{Создайте сеть согласно топологии.}
Подключите устройства, как показано в топологии, и подсоедините необходимые кабели.

\subsection{Выполните инициализацию и перезагрузку коммутаторов. }
\begin{verbatim}
    enable
    erase startup-config
    reload
\end{verbatim}

\subsection{Настройте базовые параметры каждого коммутатора.}
\begin{verbatim}
enable
config t
no ip domain lookup
hostname S1_Shestakov // S1
hostname S2 // S2
hostname S3 // S3
service password-encryption
banner motd # You must be authorizeded! #
enable secret class
line console 0
password cisco
login
line vty 0 15
password cisco
login
logging synchronous
interface range f0/5, f0/7-24, g0/1-2  // S1
interface range f0/5-17, f0/19-24, g0/1-2  // S2
interface range f0/5-17, f0/19-24, g0/1-2  // S3
shutdown
vlan 99
name Management
vlan 37
name Staff
interface f0/6  // S1
interface f0/18  // S2
interface f0/18  // S3
switchport mode access
switchport access vlan 37
no shutdown
interface vlan 99
ip address 192.168.99.11 255.255.255.0 // S1
ip address 192.168.99.12 255.255.255.0 // S2
ip address 192.168.99.13 255.255.255.0 // S3
no shutdown
end
copy running-config startup-config
\end{verbatim}

\subsection{Настройте компьютеры.}
Назначьте IP-адреса компьютерам в соответствии с таблицей адресации.
