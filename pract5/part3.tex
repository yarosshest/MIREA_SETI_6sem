\section{Настройка протокола LACP}
Протокол LACP является открытым протоколом агрегирования каналов, разработанным на базе
стандарта IEEE. В части 3 необходимо выполнить настройку канала между S1\_ФАМИЛИЯ и S2 и
канала между S2 и S3 с помощью протокола LACP. Кроме того, отдельные каналы необходимо
настроить в качестве транковых и указать native vlan, прежде чем они будут объединены в каналы
EtherChannel.



\subsection{Настройте LACP между S1\_ФАМИЛИЯ и S2}
Настройте канал между S2 и S3 как Po3, используя LACP как протокол агрегирования каналов. Канал
на S1\_ФАМЛИЛИЯ должен быть в режиме active, а канал на S2 – в режиме passive.

\begin{verbatim}
interface range f0/1-2
switchport mode trunk
switchport trunk native vlan 99
channel-group 2 mode active  // S1
channel-group 2 mode passive  // S2
no shutdown
\end{verbatim}

\subsection{Убедитесь, что порты объединены}

Какой протокол использует Po2 для агрегирования каналов?
Какие порты агрегируются для образования Po2?
Запишите команду, используемую для проверки.

Po2 использует LACP. F0/1 и F0/2 агрегируются с образованием Po2.


\img{img/png2}{Убедитесь, что порты объединены}

\subsection{Настройте LACP между S2 и S3.}
Аналогично настройте канал между S2 и S3 как Po3,
используя LACP как протокол агрегирования каналов.

\begin{verbatim}
interface range f0/3-4  // S2
interface range f0/1-2  // S3
switchport mode trunk
switchport trunk native vlan 99
channel-group 3 mode active  // S2
channel-group 3 mode passive  // S3
no shutdown
\end{verbatim}
