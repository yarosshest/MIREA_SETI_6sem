\section{Ответы на контрольные вопросы}
\subsection{Опишите назначение протокола DHCP. Назовите основные
преимущества протокола DHCP.}
Протокол динамической конфигурации узла (DHCP) динамически
назначает IP-адреса и другую информацию о конфигурации сети.

Протокол DHCPv4 является крайне полезным инструментом,
позволяющим сетевым администраторам значительно экономить время.
Использование централизованного сервера DHCP позволяет
организации управлять присвоением всех динамических IP-адресов с одного
сервера.

\subsection{Опишите принцип работы протокола DHCP. Какой тип
рассылки используется в сообщении DHCP Discover и почему?}
DHCPv4 работает по модели «клиент-сервер». Когда клиент
подключается к серверу DHCPv4, сервер присваивает или сдает ему в аренду
IPv4-адрес. Клиент с арендованным IP-адресом подключается к сети до
истечения срока аренды. Периодически клиент должен связываться с DHCP-
сервером для продления срока аренды. По истечении срока аренды сервер
DHCP возвращает адрес в пул, из которого адрес может быть повторно
получен при необходимости.

В сообщении DHCPDISCOVER используется широковещательный тип
рассылки. Поскольку во время загрузки у клиента нет верной IPv4-
информации, для связи с сервером используются широковещательные адреса
уровня 2 и уровня 3.

\subsection{Укажите основные шаги для получения IP-адреса при
использовании протокола DHCPv4. Какие основные действия
необходимо предпринять для настройки сервера DHCPv4?}
Четырехэтапном процесс получения адреса в аренду:
\begin{itemize}
    \item Обнаружение DHCP (DHCPDISCOVER).
    \item Предложение DHCP (DHCPOFFER).
    \item Запрос DHCP (DHCPREQUEST).
    \item Подтверждение DHCP (DHCPACK).
\end{itemize}

Для настройки сервера DHCPv4 Cisco IOS выполните следующие
действия:
\begin{itemize}
    \item Исключение IPv4-адресов.
    \item Определение имени пула DHCPv4.
    \item Создание пула DHCPv4.
\end{itemize}

\subsection{Какой тип рассылки используется в сообщении DHCP Request и
почему? Какие шаги используются для продления аренды IP-адреса
при использовании протокола DHCPv4?}
В сообщении DHCPREQUEST используется широковещательный тип
рассылки.

Двухэтапный процесс продления аренды с сервером DHCPv4:
\begin{itemize}
    \item DHCP \textbf{Request (DHCPREQUEST)}\\
    Перед окончанием аренды клиент отправляет сообщение
    DHCPREQUEST DHCPv4-серверу, который первоначально предложил IPv4-
    адрес, чтобы DHCPv4-сервер мог продлить срок аренды.
    \item \textbf{DHCP Acknowledgment (DHCPACK)}\\
    При получении сообщения DHCPREQUEST сервер подтверждает
    информацию об аренде ответным сообщением DHCPACK.
\end{itemize}

\subsection{Для чего необходимо использовать DHCPv4-ретрансляцию?
Перечислите варианты назначения GUA для IPv6}
Агент DHCP-ретрансляции обеспечивает ретрансляцию сообщений
DHCP между DHCP-клиентами и DHCP-серверами в различных IP-сетях.
Поскольку протокол DHCP основан на широковещательной рассылке,
пакеты этого протокола по умолчанию не проходят через маршрутизаторы.
Агент DHCP-ретрансляции получает любые широковещательные DHCP-
пакеты в одной подсети и пересылает их по заданному IP-адресу в другой
подсети.

Варианты назначения GUA: без учета состояния и с сохранением
состояния.

\subsection{Охарактеризуйте работу метода SLAAC. Какие флаги
используются в сообщении RA и что они означают?}
SLAAC – это служба без схранения состояния. Это означает, что нет
сервера, который хранит информацию о сетевых адресах, чтобы узнать,
какие IPv6-адреса используются и какие из них доступны.

Сообщение ICMPv6 RA содержит три флага:
\begin{itemize}
    \item Флаг \textbf{А} --- это флаг автонастройки адреса. Автоматическая
    конфигурация адреса без сохранения состояния (Stateless Address
    Autoconfiguration, SLAAC) для создания IPv6 GUA.
    \item Флаг \textbf{Other Configuration} (флаг \textbf{O})
    --- это флаг конфигурации
    Other. Другая информация доступна с сервера DHCPv6 без
    состояния.
    \item Флаг \textbf{Managed Address Configuration} (флаг \textbf{M})
    --- это флаг настройки управляемого (managed) адреса.
\end{itemize}

\subsection{Охарактеризуйте работу метода DHCPv6 без сохранения
состояния. Опишите методы, используемые для генерации
идентификатора интерфейса при использовании SLAAC.}
Процесс называется протокол DHCPv6 без отслеживания состояния,
поскольку сервер не поддерживает никакую информацию о состоянии
клиента, то есть список доступных и распределенных IPv6-адресов. DHCPv6-
серверы без отслеживания состояния предоставляют только параметры
конфигурации для клиента, но не выделяют IPv6-адреса.

Два метода для генерации идентификатора интерфейса (ID):
\begin{itemize}
    \item Генерация случайным образом --- 64-битный IID может быть
    случайным числом, сгенерированным операционной системой
    клиента.
    \item EUI-64 --- хост создает идентификатор интерфейса, используя
    свой 48-битный MAC-адрес и вставляет шестнадцатеричное
    значение fffe в середине адреса.
\end{itemize}

\subsection{Охарактеризуйте работу метода DHCPv6 с сохранением
состояния. Опишите основные шаги работы DHCPv6.}
DHCPv6 с отслеживанием состояния получил такое название потому,
что сервер DHCPv6 поддерживает информацию о состоянии протокола IPv6.

Шаги работы DHCPv6:
\begin{itemize}
    \item Хост отправляет сообщение RS.
    \item Маршрутизатор IPv6 отвечает сообщением RA.
    \item Хост отправляет сообщение SOLICIT DHCPv6.
    \item Сервер DHCPv6 отвечает сообщением DHCPv6 ADVERTISE.
    \item Хост отвечает серверу DHCPv6.
    \item Сервер DHCPv6 отправляет сообщение REPLY.
\end{itemize}

\subsection{Как клиент IPv6 может убедиться в уникальности своего IPv6-
адреса, полученного с помощью метода SLAAC? Какие
основные действия необходимо предпринять для настройки
сервера DHCPv6?}
Процесс обнаружения дубликатов адресов (DAD) используется хостом
для обеспечения уникальности GUA IPv6.

Для настройки и проверки маршрутизатора как DHCPv6 без
сохранения состояния сервера DHCPv6 необходимо выполнить пять шагов:
\begin{itemize}
    \item Включите маршрутизацию IPv6.
    \item Определите имя пула DHCPv6.
    \item Создайте DHCPv6-пула
    \item Привяжите пул DHCPv6 к интерфейсу.
    \item Убедитесь, что узлы получили сведения об IPv6-адресации.
\end{itemize}