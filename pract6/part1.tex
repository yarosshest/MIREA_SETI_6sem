\section*{\LARGE Выполнение работы}
\addcontentsline{toc}{section}{Выполнение работы}


\section{Создание сети и настройка основных параметров устройства}
В первой части лабораторной работы вам предстоит создать топологию сети и настроить базовые
параметры для узлов ПК и коммутаторов.

\subsection{Создание схемы адресации}
Подсеть сети 192.168.1.0/24 в соответствии со следующими требованиями:

\begin{table}[htbp]
    \centering
    \caption{Network Device Information}
    \label{tab:network_devices}
    \begin{tabular}{|l|l|l|l|l|}
        \hline
        \textbf{Device} & \textbf{Interface} & \textbf{IP Address}
        & \textbf{Subnet Mask} & \textbf{Default Gateway} \\ \hline
        R1 & G0/0/0 & 10.0.0.1 & 255.255.255.252 & N/A \\ \hline
        R1 & G0/0/1 & N/A & N/A & N/A \\ \hline
        R1 & G0/0/1.100 & 192.168.1.1 & 255.255.255.192 & N/A \\ \hline
        R1 & G0/0/1.227 & 192.168.1.65 & 255.255.255.224 & N/A \\ \hline
        R1 & G0/0/1.1000 & N/A & N/A & N/A \\ \hline
        R2 & G0/0/0 & 10.0.0.2 & 255.255.255.252 & N/A \\ \hline
        R2 & G0/0/1 & 192.168.1.97 & 255.255.255.240 & N/A \\ \hline
        S1 & VLAN 227 & 192.168.1.66 & 255.255.255.224 & 192.168.1.65 \\ \hline
        S2 & VLAN 1 & 192.168.1.98 & 255.255.255.240 & 192.168.1.97 \\ \hline
        PC & NIC & DHCP & DHCP & DHCP \\ \hline
        PC & NIC & DHCP & DHCP & DHCP \\ \hline
    \end{tabular}
\end{table}

\subsection{Выполните инициализацию и перезагрузку коммутаторов. }
\begin{verbatim}
    enable
    erase startup-config
    reload
\end{verbatim}

\subsection{Настройте базовые параметры каждого коммутатора.}
\begin{verbatim}
enable
config t
no ip domain lookup
hostname S1_Shestakov // S1
hostname S2 // S2
hostname S3 // S3
service password-encryption
banner motd # You must be authorizeded! #
enable secret class
line console 0
password cisco
login
line vty 0 15
password cisco
login
logging synchronous
interface range f0/5, f0/7-24, g0/1-2  // S1
interface range f0/5-17, f0/19-24, g0/1-2  // S2
interface range f0/5-17, f0/19-24, g0/1-2  // S3
shutdown
vlan 99
name Management
vlan 37
name Staff
interface f0/6  // S1
interface f0/18  // S2
interface f0/18  // S3
switchport mode access
switchport access vlan 37
no shutdown
interface vlan 99
ip address 192.168.99.11 255.255.255.0 // S1
ip address 192.168.99.12 255.255.255.0 // S2
ip address 192.168.99.13 255.255.255.0 // S3
no shutdown
end
copy running-config startup-config
\end{verbatim}

\subsection{Настройте компьютеры.}
Назначьте IP-адреса компьютерам в соответствии с таблицей адресации.
