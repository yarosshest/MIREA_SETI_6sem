\section*{\LARGE Выполнение работы}
\addcontentsline{toc}{section}{Выполнение работы}


\section{Создание сети и настройка основных параметров устройства}
В первой части лабораторной работы вам предстоит создать топологию сети и настроить базовые
параметры для узлов ПК и коммутаторов.

\subsection{Создание схемы адресации}
Подсеть сети 192.168.1.0/24 в соответствии со следующими требованиями:

\begin{table}[htbp]
    \centering
    \caption{Network Device Information}
    \label{tab:network_devices}
    \begin{tabular}{|l|l|l|l|l|}
        \hline
        \textbf{Device} & \textbf{Interface} & \textbf{IP Address}
        & \textbf{Subnet Mask} & \textbf{Default Gateway} \\ \hline
        R1 & G0/0/0 & 10.0.0.1 & 255.255.255.252 & N/A \\ \hline
        R1 & G0/0/1 & N/A & N/A & N/A \\ \hline
        R1 & G0/0/1.100 & 192.168.1.1 & 255.255.255.192 & N/A \\ \hline
        R1 & G0/0/1.227 & 192.168.1.65 & 255.255.255.224 & N/A \\ \hline
        R1 & G0/0/1.1000 & N/A & N/A & N/A \\ \hline
        R2 & G0/0/0 & 10.0.0.2 & 255.255.255.252 & N/A \\ \hline
        R2 & G0/0/1 & 192.168.1.97 & 255.255.255.240 & N/A \\ \hline
        S1 & VLAN 227 & 192.168.1.66 & 255.255.255.224 & 192.168.1.65 \\ \hline
        S2 & VLAN 1 & 192.168.1.98 & 255.255.255.240 & 192.168.1.97 \\ \hline
        PC & NIC & DHCP & DHCP & DHCP \\ \hline
        PC & NIC & DHCP & DHCP & DHCP \\ \hline
    \end{tabular}
\end{table}

\subsection{Создайте сеть согласно топологии.}
Подключите устройства, как показано в топологии, и подсоедините необходимые кабели.

\subsection{Произведите базовую настройку маршрутизаторов.}
\begin{verbatim}
    enable
    configure t
    hostname R1_Shestakov
    no ip domain lookup
    enable secret class
    line console 0
    password cisco
    login
    line vty 0 4
    password cisco
    login
    service password-encryption
    banner motd # You must be authorizeded! #
    copy running-config startup-config
    clock set 21:15:00 12 Mar 2024
\end{verbatim}

\subsection{Настройка маршрутизации между сетями VLAN на маршрутизаторе R1\_ФАМИЛИЯ}
Назначьте IP-адреса компьютерам в соответствии с таблицей адресации.
Активируйте интерфейс G0/0/1 на маршрутизаторе.

\begin{verbatim}
interface g0/0/1
no shutdown
exit
\end{verbatim}

Настройте подинтерфейсы для каждой VLAN
в соответствии с требованиями таблицы IPадресации.
Все субинтерфейсы используют инкапсуляцию 802.1Q
и назначаются первый полезный адрес из вычисленного пула IP-адресов.
Убедитесь, что подинтерфейсу для native VLAN не назначен IP-адрес.
Включите описание для каждого подинтерфейса.

\begin{verbatim}
interface g0/0/1.100
description Client
encapsulation dot1q 100
ip address 192.168.1.1 255.255.255.192
interface g0/0/1.203
encapsulation dot1q 203
description Management
ip address 192.168.1.65 255.255.255.224
interface g0/0/1.1000
encapsulation dot1q 1000 native
description Native
\end{verbatim}

\subsection{Настройте G0/1 на R2, затем G0/0/0
и статическую маршрутизацию для обоих маршрутизаторов}

Настройте G0/0/1 на R2 с первым IP-адресом подсети C, рассчитанным ранее.

\begin{verbatim}
interface g0/0/1
ip address 192.168.1.97 255.255.255.240
no shutdown
exit
\end{verbatim}

Настройте интерфейс G0/0/0 для каждого маршрутизатора
на основе приведенной выше таблицы IP-адресации.

\begin{verbatim}
interface g0/0/0
ip address 10.0.0.1 255.255.255.252  // R1
ip address 10.0.0.2 255.255.255.252  // R2
no shutdown
\end{verbatim}

Настройте маршрут по умолчанию на каждом маршрутизаторе,
указываемом на IP-адрес G0/0/0 на другом маршрутизаторе.

\begin{verbatim}
ip route 0.0.0.0 0.0.0.0 10.0.0.2  // R1
ip route 0.0.0.0 0.0.0.0 10.0.0.1  // R2
\end{verbatim}

Убедитесь, что статическая маршрутизация работает
с помощью отправки эхо-запроса до адреса G0/0/1 R2 от R1\_ФАМИЛИЯ.

\begin{verbatim}
ping 192.168.1.97
\end{verbatim}

Сохраните текущую конфигурацию в файл загрузочной конфигурации

\begin{verbatim}
copy running-config startup-config
\end{verbatim}

\subsection{Настройте базовые параметры каждого коммутатора}

\begin{verbatim}
hostname S1
no ip domain-lookup
enable secret class
line console 0
password cisco
login
line vty 0 4
password cisco
login
service password-encryption
banner motd # You must be authorizeded! #
exit
copy running-config startup-config
clock set 21:32:00 13 Mar 2019
\end{verbatim}

\subsection{Создайте сети VLAN на коммутаторе S1.}

Создайте необходимые VLAN на коммутаторе 1
и присвойте им имена из приведенной выше таблицы.

\begin{verbatim}
vlan 100
name Clients
vlan 200
name Management
vlan 999
name Parking\_Lot
vlan 1000
name Native
exit
\end{verbatim}

Настройте и активируйте интерфейс управления на S1 (VLAN X+200),
используя второй IP-адрес из подсети, рассчитанный ранее.
Кроме того установите шлюз по умолчанию на S1.

\begin{verbatim}
interface vlan 227
ip address 192.168.1.66 255.255.255.224
no shutdown
exit
ip default-gateway 192.168.1.65
\end{verbatim}

Настройте и активируйте интерфейс управления на S2 (VLAN 1),
используя второй IP-адрес из подсети, рассчитанный ранее.
Кроме того, установите шлюз по умолчанию на S2

\begin{verbatim}
interface vlan 1
ip address 192.168.1.98 255.255.255.240
no shutdown
exit
ip default-gateway 192.168.1.97
\end{verbatim}

Назначьте все неиспользуемые порты S1 VLAN Parking\_Lot,
настройте их для статического режима доступа
и административно деактивируйте их.
На S2 административно деактивируйте все неиспользуемые порты.

\begin{verbatim}
interface range f0/1 - 4, f0/7 - 24, g0/1 - 2  // S1
interface range f0/1 - 4, f0/6 - 17, f0/19 - 24, g0/1 - 2  // S2
switchport mode access
switchport access vlan 999  // S1
shutdown
exit
\end{verbatim}

\subsection{Назначьте сети VLAN соответствующим интерфейсам коммутатора}

Назначьте используемые порты соответствующей VLAN
(указанной в таблице VLAN выше)
и настройте их для режима статического доступа.

\begin{verbatim}
// S1
interface f0/6
switchport mode access
switchport access vlan 100
\end{verbatim}


Убедитесь, что VLAN назначены на правильные интерфейсы.

\begin{verbatim}
// S1
show vlan brief
\end{verbatim}

\textbf{Почему интерфейс F0/5 указан в VLAN 1?}\\
Порт 5 находится в VLAN по умолчанию и не был настроен как магистраль 802.1Q.

\subsection{Вручную настройте интерфейс S1 F0/5 в качестве транка 802.1Q}

Измените режим порта коммутатора,
чтобы принудительно создать магистральный канал.

\begin{verbatim}
interface f0/5
switchport mode trunk
\end{verbatim}

В рамках конфигурации транкового канала установите
для native VLAN значение 1000.

\begin{verbatim}
switchport trunk native vlan 1000
\end{verbatim}

В качестве другой части конфигурации магистрали укажите, что VLAN 100, X+200 и 1000 могут проходить по транковому каналу.

\begin{verbatim}
switchport trunk allowed vlan 100,227,1000
\end{verbatim}

Сохраните текущую конфигурацию в файл загрузочной конфигурации.

\begin{verbatim}
copy running-config startup-config
\end{verbatim}

Проверьте состояние транкового канала.

\begin{verbatim}
show interfaces trunk
\end{verbatim}

\textbf{Какой IP-адрес был бы у ПК,
    если бы он был подключен к сети с помощью DHCP?}\\
Они будут самонастраиваться с помощью автоматического
частного IP-адреса (APIPA) в диапазоне 169.254.x.x.