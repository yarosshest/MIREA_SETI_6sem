\section{Настройка и проверка двух серверов DHCPv4 на R1\_ФАМИЛИЯ}

\subsection{Настройте R1\_ФАМИЛИЯ с пулами DHCPv4
для двух поддерживаемых подсетей}

\begin{verbatim}
ip dhcp excluded-address 192.168.1.1 192.168.1.5
ip dhcp pool R1_Client_LAN
network 192.168.1.0 255.255.255.192
domain-name ccna-lab.com
default-router 192.168.1.1
lease 2 12 30

ip dhcp excluded-address 192.168.1.97 192.168.1.101
ip dhcp pool R2_Client_LAN
network 192.168.1.96 255.255.255.240
default-router 192.168.1.97
domain-name ccna-lab.com
lease 2 12 30
\end{verbatim}

\subsection{Сохраните конфигурацию}
Сохраните текущую конфигурацию в файл загрузочной конфигурации.
Закройте окно настройки.

\begin{verbatim}
copy running-config startup-config
\end{verbatim}

\subsection{Проверка конфигурации сервера DHCPv4}

Чтобы просмотреть сведения о пуле,
выполните команду \texttt{show ip dhcp pool}.

Выполните команду \texttt{show ip dhcp binding}
для проверки установленных назначений адресов DHCP.
Выполните команду \textbf{show ip dhcp server statistics}
для проверки сообщений DHCP.

\textbf{NOT WORK?}

\subsection{Попытка получить IP-адрес от DHCP на PC-A}

Из командной строки компьютера PC-A выполните команду ipconfig /all.

После завершения процесса обновления выполните команду ipconfig для просмотра новой информации об IP-адресе.

Проверьте подключение с помощью эхо-запроса
на IP-адрес интерфейса R1\_ФАМИЛИЯ G0/0/1.

\begin{verbatim}
ping 192.168.1.1
\end{verbatim}

\subsection{Настройка и проверка DHCP-ретрансляции на R2}

\subsection{\\Настройка R2 в качестве агента DHCP-ретрансляции
для локальной сети на G0/0/1}

Настройте команду ip helper-address на G0/0/1,
указав IP-адрес G0/0/0 R1\_ФАМИЛИЯ.

\begin{verbatim}
// R2
interface g0/0/1
ip helper-address 10.0.0.1
\end{verbatim}

Сохраните конфигурацию.

\begin{verbatim}
copy running-config startup-config
\end{verbatim}

\subsection{Попытка получить IP-адрес от DHCP на PC-B}

Из командной строки компьютера PC-B выполните команду \texttt{ipconfig /all}.

После завершения процесса обновления выполните команду\texttt{ipconfig}
для просмотра новой информации об IP-адресе.

Проверьте подключение с помощью эхо-запроса
на IP-адрес интерфейса R1\_ФАМИЛИЯ G0/0/1.

\begin{verbatim}
ping 192.168.1.1
\end{verbatim}

Выполните \texttt{show ip dhcp binding} для R1\_ФАМИЛИЯ =
для проверки назначений адресов в DHCP.


Выполните команду \texttt{show ip dhcp server statistics}
для проверки сообщений DHCP.

\textbf{NOT WORK?}