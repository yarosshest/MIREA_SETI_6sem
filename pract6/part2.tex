\section{Настройка протокола PAgP}
Протокол PAgP является проприетарным протоколом агрегирования каналов Cisco. В части 2 вам
предстоит настроить канал между S1_ФАМИЛИЯ и S3 с использованием протокола PAgP.

\subsection{Настройте PAgP на S1_ФАМИЛИЯ и S3.}
\begin{verbatim}
config t
interface range f0/3-4
channel-group 1 mode desirable  // S1
channel-group 1 mode auto  // S3
no shutdown
\end{verbatim}

\subsection{Проверьте конфигурации на портах}
\begin{verbatim}
show interfaces f0/3 switchport
\end{verbatim}

\img{img/png}{Проверьте конфигурации на портах}

\subsection{Убедитесь, что порты объединены.}
\begin{verbatim}
show etherchannel summary
\end{verbatim}

\img{img/png1}{etherchannel summary}

\textbf{Что означают флаги «SU» и «P» в сводных данных по Ethernet?}\\
Флаг P указывает, что порты объединены в порт-канал.\\
Флаг S указывает, что порт-канал является EtherChannel уровня 2.\\
Флаг U указывает, что EtherChannel используется.\\

\subsection{Настройте транковые порты.}

\textbf{S1 S3}

\begin{verbatim}
config t
interface port-channel 1
switchport mode trunk
switchport trunk native vlan 99
\end{verbatim}

\subsection{Убедитесь в том, что порты настроены в качестве транковых}

Выполните команды \verb|show running-startup| идентификатор-интерфейса
на S1\_ФАМИЛИЯ и S3.
Какие команды включены в список для интерфейсов F0/3 и F0/4
на обоих коммутаторах?

\begin{verbatim}
switchport trunk native vlan 99
switchport mode trunk
\end{verbatim}

Сравните результаты с текущей конфигурацией для интерфейса Po1.
Запишите наблюдения.

Команды, относящиеся к конфигурации таранков, одинаковы.
Когда команды таранков были применены к EtherChannel,
команды также повлияли на отдельные ссылки в пакете.

\begin{verbatim}
!
interface Port-channel1
 switchport trunk native vlan 99
 switchport mode trunk
!
interface FastEthernet0/1
!
interface FastEthernet0/2
!
interface FastEthernet0/3
 switchport trunk native vlan 99
 switchport mode trunk
 channel-group 1 mode desirable
!
\end{verbatim}

Выполните команды \verb|show interfaces trunk| и \verb|show spanning-tree|
на S1\_ФАМИЛИЯ и S3.
Какой транковый порт включен в список?
Какая используется сеть native VLAN?
Какой вывод можно сделать на основе выходных данных?

Указанный таранковый порт - Po1.
Собственная VLAN - 99.
После объединения ссылок в некоторых командах отображается только
агрегированный интерфейс.

Какие значения стоимости и приоритета порта
для агрегированного канала отображены
в выходных данных команды show spanning-tree?

Стоимость порта для Po1 равна 12, а приоритет порта равен 128.