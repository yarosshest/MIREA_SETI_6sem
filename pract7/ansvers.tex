\section{Ответы на контрольные вопросы}

\subsection{Для чего необходимо обеспечить безопасность портов
коммутатора? Что произойдет, если к порту с включенной
безопасностью подключают более одного устройства и почему?}

Обеспечение безопасности портов коммутатора имеет ряд важных
причин:

\begin{itemize}
    \item защита от несанкционированного доступа;
    \item защита от атак;
    \item контроль трафика;
    \item повышение устойчивости сети.
\end{itemize}

Если активный порт настроен с помощью команды switchport port-
security и к этому порту подключено более одного устройства, порт перейдет
в состояние error-disabled.

\subsection{Какое минимальное и максимальное количество MAC-адресов
может быть разрешено на одном порту коммутатора? Опишите
все существующие способы изучения MAC-адресов на
коммутаторе}

Для определения максимального числа MAC адресов, разрешенных для
конкретного порта, используется команда:

\begin{verbatim}
Switch(config-if)# switchport port-security maximum ?
\end{verbatim}

Значение безопасности порта по умолчанию равно 1. Максимальное
количество защищенных MAC-адресов, которые можно настроить, зависит
от коммутатора и IOS.

Коммутатор может быть настроен на изучение MAC-адресов на
защищенном порту одним из трех способов:

\begin{itemize}
    \item Вручную
    \item Динамически изученный
    \item Динамически изученный sticky MAC адрес
\end{itemize}

\subsection{Опишите существующие типы устаревания безопасности порта.
Каким образом можно активировать отключенный по ошибке
порт коммутатора?}

Для каждого порта поддерживается два типа старения:

\begin{itemize}
    \item Абсолютный --- защищенные адреса порта удаляются по истечении
    указанного времени устаревания.
    \item По таймеру неактивности --- безопасные адреса на порту
    удаляются, только если они неактивны в течение указанного
    времени.
\end{itemize}

Когда защищенный порт находится в состоянии отключения по
ошибке, администратор должен повторно включить его, используя сначала
команду shutdown, затем используя команду no shutdown, чтобы сделать
порт работоспособным.

\subsection{Дайте характеристику режимам нарушения безопасности порта.
В чем заключается опасность включенного протокола
согласования DTP?}

Режимы нарушения безопасности порта:

\begin{itemize}
    \item Защита
    \item Ограничение
    \item Выключение
\end{itemize}

Подмена сообщений DTP от атакующего хоста, чтобы заставить
коммутатор войти в режим транкинга. Отсюда злоумышленник может
отправлять трафик, помеченный целевой VLAN, а затем коммутатор
доставляет пакеты в пункт назначения.

\subsection{Опишите суть технологии DHCP Snooping. Для чего может
понадобиться динамиеская проверка ARP?}

DHCP Snooping определяет, является ли DHCP сообщение от
административно сконфигурированных доверенного или ненадежного
источника. Затем он фильтрует сообщения DHCP и объем DHCP трафика из
ненадежных источников.\par
В типичной атаке ARP злоумышленник может отправлять
незапрошенные ответы ARP другим узлам в подсети с MAC-адресом
субъекта угрозы и IP-адресом шлюза по умолчанию. Чтобы предотвратить
подделку ARP и вызванное ею отравление ARP, коммутатор должен
обеспечить передачу только действительных запросов и ответов ARP.
Динамическая проверка ARP (DAI) требует отслеживания DHCP и
помогает предотвратить атаки ARP.

\subsection{Перечислите рекомендации по настройке портов с помощью
динамической проверки ARP. Почему необходимо включать
функции BPDU Guard И PortFast?}

Рекомендации по настройке портов с помощью DAI:

\begin{itemize}
    \item Включить отслеживание DHCP на глобальном уровне.
    \item Включите отслеживание DHCP на выбранных VLAN.
    \item Включите DAI на выбранных VLAN.
    \item Настроить доверенные интерфейсы для отслеживания DHCP и
    проверки ARP.
\end{itemize}

Чтобы нейтрализовать атаки манипуляций с протоколом STP,
используйте средства защиты PortFast и Bridge Protocol Data Unit (BPDU).

\subsection{Какие шаги необходимо предпринять для устранения угрозы
VLAN Hopping?}

Шаги, чтобы нейтрализовать атаки VLAN Hopping:
Отключите согласование DTP (автоматические магистральные
каналы) на немагистральных портах с помощью команды
интерфейсной настройки switchport mode access.
Отключите неиспользуемые порты и назначьте их
неиспользуемой VLAN.\par
Вручную включите магистральный канал на магистральном
порту с помощью команды интерфейсной настройки switchport
mode trunk.\par
Отключите согласование DTP (автоматические магистральные
каналы) на немагистральных портах с помощью команды
интерфейсной настройки switchport mode access.
Установите для native VLAN, VLAN, отличную от VLAN 1, с
помощью команды \verb|switchport trunk native vlan vlan_number command|.

\subsection{Что рекомендуется сделать при использовании сети native
VLAN? Какие два типа портов коммутаторов используются на
коммутаторах Cisco в составе средств защиты от атак DHCP
Snooping?}

Установить для native VLAN, VLAN, отличную от VLAN 1, с помощью
команды \verb|switchport trunk native vlan vlan_number command|.\par
Доверенные порты DHCP --- порты коммутатора, подключенные
к восходящим DHCP-серверам.\par
Недоверенные порты --- порты коммутатора, которые служат для
подключения узлов и которые не должны использоваться для
сообщений сервера DHCP.

\subsection{Почему устройства уровня 2 считаются самым слабым звеном в
инфраструктуре безопасности компании? Где хранятся
динамически определяемые MAC-адреса, когда включена
функция sticky learning?}

Устройства уровня 2 считаются одними из наиболее уязвимых звеньев
в инфраструктуре безопасности компании по нескольким причинам:

\begin{itemize}
    \item Отсутствие фильтрации трафика.
    \item Отсутствие средств защиты на уровне 2.
    \item Отсутствие шифрования.
\end{itemize}

Когда включена функция sticky learning, динамически определяемые
MAC-адреса хранятся в таблице MAC-адресов коммутатора.