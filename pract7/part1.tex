\section*{\LARGE Выполнение работы}
\addcontentsline{toc}{section}{Выполнение работы}


\section{Настройка основного сетевого устройства}
В первой части лабораторной работы вам предстоит создать топологию сети и настроить базовые
параметры для узлов ПК и коммутаторов.

\subsection{Создайте сеть}
\begin{itemize}
    \item Создайте сеть согласно топологии.
    \item Инициализация устройств
\end{itemize}

\subsection{Настройте маршрутизатор R1\_ФАМИЛИЯ}
\begin{verbatim}
    enable
    configure terminal
    hostname R1_shetakov
    no ip domain lookup
    ip dhcp excluded-address 192.168.37.1 192.168.37.9
    ip dhcp excluded-address 192.168.37.201 192.168.37.202
    !
    ip dhcp pool Students
    network 192.168.37.0 255.255.255.0
    default-router 192.168.37.1
    domain-name CCNA2.Lab-7
    !
    interface Loopback0
    ip address 10.10.1.1 255.255.255.0
    !
    interface GigabitEthernet0/0/1
    description Link to S1
    ip dhcp relay information trusted
    ip address 192.168.37.1 255.255.255.0
    no shutdown
    !
    line con 0
    logging synchronous
    exec-timeout 0 0
\end{verbatim}

\subsection{Настройка и проверка основных параметров коммутатора}
Настройте имя хоста для коммутаторов S1 и S2.

\begin{verbatim}
config t
hostname S1  // S1
hostname S2  // S2
\end{verbatim}

Запретите нежелательный поиск в DNS.

\begin{verbatim}
no ip domain-lookup
\end{verbatim}

Настройте описания интерфейса для портов, которые используются в S1 и S2.

\begin{verbatim}
// S1
interface f0/1
description Link to S2
interface f0/5
description Link to R1
interface f0/6
description Link to PC-A

// S2
interface f0/1
description Link to S1
interface f0/18
description Link to PC-B
\end{verbatim}

Установите для шлюза по умолчанию для VLAN управления значение 192.168.37.1
на обоих коммутаторах.

\begin{verbatim}
ip default-gateway 192.168.37.1
\end{verbatim}

