\section*{\LARGE Выполнение работы}
\addcontentsline{toc}{section}{Выполнение работы}


\section{Настройка основного сетевого устройства}
В первой части лабораторной работы вам предстоит создать топологию сети и настроить базовые
параметры для узлов ПК и коммутаторов.

\subsection{Создайте сеть}
\begin{itemize}
    \item Создайте сеть согласно топологии.
    \item Инициализация устройств
\end{itemize}

\subsection{Настройте маршрутизатор R1\_ФАМИЛИЯ}
\begin{verbatim}
    enable
    configure terminal
    hostname R1_shetakov
    no ip domain lookup
    ip dhcp excluded-address 192.168.37.1 192.168.37.9
    ip dhcp excluded-address 192.168.37.201 192.168.37.202
    !
    ip dhcp pool Students
    network 192.168.37.0 255.255.255.0
    default-router 192.168.37.1
    domain-name CCNA2.Lab-7
    !
    interface Loopback0
    ip address 10.10.1.1 255.255.255.0
    !
    interface GigabitEthernet0/0/1
    description Link to S1
    ip dhcp relay information trusted
    ip address 192.168.37.1 255.255.255.0
    no shutdown
    !
    line con 0
    logging synchronous
    exec-timeout 0 0
\end{verbatim}

\subsection{Выполните инициализацию и перезагрузку коммутаторов. }
\begin{verbatim}
    enable
    erase startup-config
    reload
\end{verbatim}

\subsection{Настройте базовые параметры каждого коммутатора.}
\begin{verbatim}
enable
config t
no ip domain lookup
hostname S1_Shestakov // S1
hostname S2 // S2
hostname S3 // S3
service password-encryption
banner motd # You must be authorizeded! #
enable secret class
line console 0
password cisco
login
line vty 0 15
password cisco
login
logging synchronous
interface range f0/5, f0/7-24, g0/1-2  // S1
interface range f0/5-17, f0/19-24, g0/1-2  // S2
interface range f0/5-17, f0/19-24, g0/1-2  // S3
shutdown
vlan 99
name Management
vlan 37
name Staff
interface f0/6  // S1
interface f0/18  // S2
interface f0/18  // S3
switchport mode access
switchport access vlan 37
no shutdown
interface vlan 99
ip address 192.168.99.11 255.255.255.0 // S1
ip address 192.168.99.12 255.255.255.0 // S2
ip address 192.168.99.13 255.255.255.0 // S3
no shutdown
end
copy running-config startup-config
\end{verbatim}

\subsection{Настройте компьютеры.}
Назначьте IP-адреса компьютерам в соответствии с таблицей адресации.
