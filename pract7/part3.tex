\section{Настройки безопасности коммутатора.}

\subsection{Релизация магистральных соединений 802.1Q.}

Настройте все магистральные порты Fa0/1 на обоих коммутаторах
для использования VLAN 333 в качестве native VLAN.

\begin{verbatim}
interface f0/1
switchport mode trunk
switchport trunk native vlan 333
\end{verbatim}

Убедитесь, что режим транкинга успешно настроен на всех коммутаторах
с помощью команды \texttt{show interface trunk} на обоих коммутаторах.

Отключить согласование DTP F0/1 на S1 и S2.

\begin{verbatim}
interface f0/1
switchport nonegotiate
\end{verbatim}

Проверьте с помощью команды \texttt{show interfaces}.

\subsection{Настройка портов доступа}

На S1 настройте F0/5 и F0/6 в качестве портов доступа и свяжите их с VLAN 37.

\begin{verbatim}
interface range f0/5 - 6
switchport mode access
switchport access vlan 37
\end{verbatim}

На S2 настройте порт доступа Fa0/18 и свяжите его с VLAN 37.

\begin{verbatim}
interface f0/18
switchport mode access
switchport access vlan 37
\end{verbatim}

\subsection{Безопасность неиспользуемых портов коммутатора}

На S1 и S2 переместите неиспользуемые порты из VLAN 1 в VLAN 999
и отключите неиспользуемые порты.

\begin{verbatim}
interface range f0/2-4 , f0/7-24, g0/1-2  // S1
interface range f0/2-17 , f0/19-24, g0/1-2  // S2
switchport mode access
switchport access vlan 999
shutdown
\end{verbatim}

Убедитесь, что неиспользуемые порты отключены и связаны с VLAN 999,
введя команду \texttt{show interfaces status}.

\subsection{Документирование и реализация функций безопасности порта}

Интерфейсы F0/6 на S1 и F0/18 на S2 настроены как порты доступа.
На этом шаге вы также настроите безопасность портов
на этих двух портах доступа.

На S1 введите команду\texttt{show port-security interface f0/6}
для отображения настроек по умолчанию безопасности порта для интерфейса F0/6.

На S1 включите защиту порта на F0/6 со следующими настройками:

\begin{itemize}
    \item Максимальное количество записей MAC-адресов: 3
    \item Режим безопасности: restrict
    \item Aging time: 60 мин.
    \item Aging type: неактивный
\end{itemize}

\begin{verbatim}
interface f0/6
switchport port-security
switchport port-security maximum 3
switchport port-security violation restrict
switchport port-security aging time 60
switchport port-security aging type inactivity
\end{verbatim}

Проверьте настройки защиты порта (port-security) на S1 для интерфейса F0/6.
Далее просмотрите выходные данные команды \texttt{show port-security address}.

\begin{verbatim}
show port-security interface f0/6
\end{verbatim}

\begin{verbatim}
show port-security address
\end{verbatim}


Включите безопасность порта для F0/18 на S2.
Настройте каждый активный порт доступа таким образом,
чтобы он автоматически добавлял адреса МАС, изученные на этом порту,
в текущую конфигурацию.

\begin{verbatim}
interface f0/18
switchport port-security
switchport port-security mac-address sticky
\end{verbatim}

Настройте следующие параметры безопасности порта на S2 F0/18:

\begin{itemize}
    \item Максимальное количество записей MAC-адресов: 2
    \item Тип безопасности: Protect
    \item Aging time: 60 мин.
\end{itemize}

\begin{verbatim}
interface f0/18
switchport port-security aging time 60
switchport port-security maximum 2
switchport port-security violation protect
\end{verbatim}

Проверьте настройки защиты порта (port-security)
на S2 для интерфейса F0/18. Далее просмотрите выходные
данные команды \texttt{show port-security address}.

\begin{verbatim}
show port-security interface f0/18
\end{verbatim}

\begin{verbatim}
show port-security address
\end{verbatim}

\subsection{Реализовать безопасность DHCP snooping}

На S2 включите DHCP snooping и настройте DHCP snooping во VLAN X+10.

\begin{verbatim}
ip dhcp snooping
ip dhcp snooping vlan 37
\end{verbatim}

Настройте магистральные порты на S2 как доверенные порты.

\begin{verbatim}
interface f0/1
ip dhcp snooping trust
\end{verbatim}

Ограничьте ненадежный порт Fa0/18 на S2 пятью DHCP-пакетами в секунду.

\begin{verbatim}
interface f0/18
ip dhcp snooping limit rate 5
\end{verbatim}

Проверьте DHCP Snooping на S2 с помощью команды show ip dhcp snooping.

\begin{verbatim}
show ip dhcp snooping
\end{verbatim}


В командной строке на PC-B освободите, а затем обновите IP-адрес.

\begin{verbatim}
C:\Users\Student> ipconfig /release
C:\Users\Student> ipconfig /renew
\end{verbatim}


Проверьте привязку отслеживания DHCP
с помощью команды \texttt{show ip dhcp snooping binding}.



\subsection{Реализация PortFast и BPDU Guard}

Настройте PortFast на всех портах доступа,
которые используются на обоих коммутаторах.

\begin{verbatim}
interface range f0/5 - 6  // S1
interface f0/18  // S2
spanning-tree portfast
\end{verbatim}

Включите защиту BPDU на портах доступа VLAN 13 для S1 и S2,
подключенных к PC-A и PC-B.

\begin{verbatim}
interface f0/6  // S1
interface f0/18  // S2
spanning-tree bpduguard enable
\end{verbatim}

Убедитесь, что защита BPDU и PortFast включены на соответствующих портах
с помощью команды \texttt{show spanning-tree interface f0/6 detail}.

\begin{verbatim}
show spanning-tree interface f0/6 detail
\end{verbatim}

\subsection{Проверьте наличие сквозного подключения}

Отправьте эхо-запрос между всеми устройствами в таблице IP-адресации.

\section{Ответы на контрольные вопросы}

\subsection{Для чего необходимо обеспечить безопасность портов
коммутатора? Что произойдет, если к порту с включенной
безопасностью подключают более одного устройства и почему?}

Обеспечение безопасности портов коммутатора имеет ряд важных
причин:

\begin{itemize}
    \item защита от несанкционированного доступа;
    \item защита от атак;
    \item контроль трафика;
    \item повышение устойчивости сети.
\end{itemize}

Если активный порт настроен с помощью команды switchport port-
security и к этому порту подключено более одного устройства, порт перейдет
в состояние error-disabled.

\subsection{Какое минимальное и максимальное количество MAC-адресов
может быть разрешено на одном порту коммутатора? Опишите
все существующие способы изучения MAC-адресов на
коммутаторе}

Для определения максимального числа MAC адресов, разрешенных для
конкретного порта, используется команда:

\begin{verbatim}
Switch(config-if)# switchport port-security maximum ?
\end{verbatim}

Значение безопасности порта по умолчанию равно 1. Максимальное
количество защищенных MAC-адресов, которые можно настроить, зависит
от коммутатора и IOS.

Коммутатор может быть настроен на изучение MAC-адресов на
защищенном порту одним из трех способов:

\begin{itemize}
    \item Вручную
    \item Динамически изученный
    \item Динамически изученный sticky MAC адрес
\end{itemize}

\subsection{Опишите существующие типы устаревания безопасности порта.
Каким образом можно активировать отключенный по ошибке
порт коммутатора?}

Для каждого порта поддерживается два типа старения:

\begin{itemize}
    \item Абсолютный --- защищенные адреса порта удаляются по истечении
    указанного времени устаревания.
    \item По таймеру неактивности --- безопасные адреса на порту
    удаляются, только если они неактивны в течение указанного
    времени.
\end{itemize}

Когда защищенный порт находится в состоянии отключения по
ошибке, администратор должен повторно включить его, используя сначала
команду shutdown, затем используя команду no shutdown, чтобы сделать
порт работоспособным.

\subsection{Дайте характеристику режимам нарушения безопасности порта.
В чем заключается опасность включенного протокола
согласования DTP?}

Режимы нарушения безопасности порта:

\begin{itemize}
    \item Защита
    \item Ограничение
    \item Выключение
\end{itemize}

Подмена сообщений DTP от атакующего хоста, чтобы заставить
коммутатор войти в режим транкинга. Отсюда злоумышленник может
отправлять трафик, помеченный целевой VLAN, а затем коммутатор
доставляет пакеты в пункт назначения.

\subsection{Опишите суть технологии DHCP Snooping. Для чего может
понадобиться динамиеская проверка ARP?}

DHCP Snooping определяет, является ли DHCP сообщение от
административно сконфигурированных доверенного или ненадежного
источника. Затем он фильтрует сообщения DHCP и объем DHCP трафика из
ненадежных источников.\par
В типичной атаке ARP злоумышленник может отправлять
незапрошенные ответы ARP другим узлам в подсети с MAC-адресом
субъекта угрозы и IP-адресом шлюза по умолчанию. Чтобы предотвратить
подделку ARP и вызванное ею отравление ARP, коммутатор должен
обеспечить передачу только действительных запросов и ответов ARP.
Динамическая проверка ARP (DAI) требует отслеживания DHCP и
помогает предотвратить атаки ARP.

\subsection{Перечислите рекомендации по настройке портов с помощью
динамической проверки ARP. Почему необходимо включать
функции BPDU Guard И PortFast?}

Рекомендации по настройке портов с помощью DAI:

\begin{itemize}
    \item Включить отслеживание DHCP на глобальном уровне.
    \item Включите отслеживание DHCP на выбранных VLAN.
    \item Включите DAI на выбранных VLAN.
    \item Настроить доверенные интерфейсы для отслеживания DHCP и
    проверки ARP.
\end{itemize}

Чтобы нейтрализовать атаки манипуляций с протоколом STP,
используйте средства защиты PortFast и Bridge Protocol Data Unit (BPDU).

\subsection{Какие шаги необходимо предпринять для устранения угрозы
VLAN Hopping?}

Шаги, чтобы нейтрализовать атаки VLAN Hopping:
Отключите согласование DTP (автоматические магистральные
каналы) на немагистральных портах с помощью команды
интерфейсной настройки switchport mode access.
Отключите неиспользуемые порты и назначьте их
неиспользуемой VLAN.\par
Вручную включите магистральный канал на магистральном
порту с помощью команды интерфейсной настройки switchport
mode trunk.\par
Отключите согласование DTP (автоматические магистральные
каналы) на немагистральных портах с помощью команды
интерфейсной настройки switchport mode access.
Установите для native VLAN, VLAN, отличную от VLAN 1, с
помощью команды \verb|switchport trunk native vlan vlan_number command|.

\subsection{Что рекомендуется сделать при использовании сети native
VLAN? Какие два типа портов коммутаторов используются на
коммутаторах Cisco в составе средств защиты от атак DHCP
Snooping?}

Установить для native VLAN, VLAN, отличную от VLAN 1, с помощью
команды \verb|switchport trunk native vlan vlan_number command|.\par
Доверенные порты DHCP --- порты коммутатора, подключенные
к восходящим DHCP-серверам.\par
Недоверенные порты --- порты коммутатора, которые служат для
подключения узлов и которые не должны использоваться для
сообщений сервера DHCP.

\subsection{Почему устройства уровня 2 считаются самым слабым звеном в
инфраструктуре безопасности компании? Где хранятся
динамически определяемые MAC-адреса, когда включена
функция sticky learning?}

Устройства уровня 2 считаются одними из наиболее уязвимых звеньев
в инфраструктуре безопасности компании по нескольким причинам:

\begin{itemize}
    \item Отсутствие фильтрации трафика.
    \item Отсутствие средств защиты на уровне 2.
    \item Отсутствие шифрования.
\end{itemize}

Когда включена функция sticky learning, динамически определяемые
MAC-адреса хранятся в таблице MAC-адресов коммутатора.